%% The first command in your LaTeX source must be the \documentclass command.
%%
%% Options:
%% twocolumn : Two column layout.
%% hf: enable header and footer.
\documentclass[
% twocolumn,
% hf,
]{ceurart}

%%
%% One can fix some overfulls
\sloppy

%%
%% Minted listings support 
%% Need pygment <http://pygments.org/> <http://pypi.python.org/pypi/Pygments>
\usepackage{listings}
\usepackage{amsmath}
\usepackage{multirow}
\usepackage{graphicx}
\usepackage{caption}
\usepackage{subcaption}
\usepackage{threeparttable}
\graphicspath{ {./img/} }
%% auto break lines
\lstset{breaklines=true}

% \newcommand{\newrev}[1]{{\textbf{{\color{blue}#1}}}}
\newcommand{\newrev}[1]{{{#1}}}

\newcommand{\CellWithForceBreak}[2][c]{
\begin{tabular}[#1]{@{}c@{}}#2\end{tabular}}

%%
%% end of the preamble, start of the body of the document source.
\begin{document}

%%
%% Rights management information.
%% CC-BY is default license.
\copyrightyear{2022}
\copyrightclause{Copyright for this paper by its authors.
  Use permitted under Creative Commons License Attribution 4.0
  International (CC BY 4.0).}

%%
%% This command is for the conference information
\conference{To be decided}

%%
%% The "title" command
\title{An Analytical Framework for Knowledge Wealth of Knowledge Graphs}

% \tnotemark[1]
% \tnotetext[1]{You can use this document as the template for preparing your
%   publication. We recommend using the latest version of the ceurart style.}

%%
%% The "author" command and its associated commands are used to define
%% the authors and their affiliations.
\author[1]{TBD}[%
% orcid=0000-0002-0877-7063,
email=tbd@ui.ac.id,
% url=https://yamadharma.github.io/,
]
% \cormark[1]
% \fnmark[1]
\author[1]{TBD}[%
% orcid=0000-0002-0877-7063,
email=tbd@ui.ac.id,
% url=https://yamadharma.github.io/,
]
% \cormark[1]
% \fnmark[1]
\address[1]{Faculty of Computer Science, Universitas Indonesia, Depok, Indonesia}

\author[2]{TBD}[%
% orcid=0000-0001-7116-9338,
email=tbd@catapa.com,
% url=https://kmitd.github.io/ilaria/,
]
% \fnmark[1]
\address[2]{CATAPA, Jakarta, Indonesia}

\author[3]{TBD}[%
% orcid=0000-0002-9421-8566,
email=tbd@gdplabs.id,
% url=http://conceptbase.sourceforge.net/mjf/,
]
% \fnmark[1]
\address[3]{GDP Labs, Jakarta, Indonesia}

%% Footnotes
% \cortext[1]{Corresponding author.}
% \fntext[1]{These authors contributed equally.}

%%
%% The abstract is a short summary of the work to be presented in the
%% article.
\begin{abstract}
Along with the rapid development of data volumes, the need for machine-readable data is inevitable. As a result, the use of knowledge graph data structures becomes more popular. With its development, quality aspects of a knowledge graph need to be considered, one of which is knowledge wealth: the amount of information contained in a knowledge graph. A high level of knowledge wealth in a knowledge graph may indicate the high quality of a knowledge graph; conversely, a low level of knowledge wealth can be a sign of poor quality of a knowledge graph. However, there is no formal way to define knowledge wealth and how to measure and analyze it. This study proposes a framework to analyze knowledge wealth and the level of knowledge imbalance in the RDF knowledge graph by seeing how the knowledge wealth of an entity class is spread over the knowledge graph using statistical measures and visualization. In addition, this framework also helps to identify entity groups based on the level of wealth in their class, finds the best theoretical distribution that fits best to knowledge wealth distribution, performs clustering on classes based on the shape of knowledge wealth distribution, and detects bias in a knowledge graph. To evaluate this framework, some use cases were conducted on several classes on Wikidata. It is hoped that the results of this study can assist in researching knowledge wealth in the knowledge graph and be used to optimize the efforts of editing and developing knowledge graph projects by the contributors.
\end{abstract}

%%
%% Keywords. The author(s) should pick words that accurately describe
%% the work being presented. Separate the keywords with commas.
\begin{keywords}
  Knowledge graphs \sep
  knowledge wealth \sep
  knowledge imbalance analysis
\end{keywords}

%%
%% This command processes the author and affiliation and title
%% information and builds the first part of the formatted document.
\maketitle


%% include motivating scenario
\section{Introduction}

% To do:
% 1. American researchers: Yang punya property 1, 2, 3, dst itu mereka punya property apa?
% 2. [DONE] Outlier & sample size effect thd gini?
% 3. [DONE] Cek perbedaan 2 entitas di Introduction (pakai query difference) --> + Introduction

% To do (part 2):
% 4. Arsitektural: Library, Flow, etc
% 5. [DONE] Sitasi introduction
% 6. [DONE] Formatting introduction
% 7. Revisi penulisan bias wikidata

% %% include motivating scenario

% (tambahkan 2 paragraf pembukaan)
% distinct - minus = apa yang ada di NYC tapi ga ada di Batam
% Kasih contoh:
% 1. Entitas dengan tipe sama
% 2. Tunjukkan adanya singevalue vs multivalue
% 3. Tunjukkan adanya presence vs absence
% 4. Tunjukkan adanya incoming vs outgoing (incoming itu ga kalah penting, bisa jadi outgoing rendah tapi incoming nya banyak di refer entitas lain)
% 5. Tunjukkan adanya prop type

% contoh lain:
% 1. yang incomingnya banyak
% 2. yang external ID banyak

% Poin:
% 1. ...
% 2. ...

\begin{figure}[!htbp]
    \centering
    % \includegraphics[scale=.5]{Gini - Pure Literal}
    \makebox[\textwidth][c]{\includegraphics[width=1\textwidth]{Introduction - Wikidata 2}}%
    \caption{Data on Wikidata about the Romani people, the Minangkabau, and the Ambonese} \label{fig:intro-wikidata}
\end{figure}

Wealth is the abundance of valuable possessions or money, or the state of having this abundance (Oxford English Dictionary). In economics, wealth is defined as the net value of all assets owned by individuals or households, calculated as total assets minus liabilities, and used to assess economic inequality (Saez \& Zucman, 2016). When considering individual (human) wealth, the most common measure is net worth. On the other hand, in knowledge graphs (KGs), wealth can be defined as the amount of information (in terms of properties or links) an entity possesses.

\autoref{fig:intro-wikidata} shows three Wikidata entities of the type ethnic group: the Romani people, the Minangkabau, and the Ambonese, along with some information about them. For example, the entity Romani people includes information such as its class, flag, native languages, population, and associated Freebase ID. We can observe that a type of information can have a single value, such as the property \textit{flag} for each entity, which is exactly one (or zero, if the information does not exist), or multiple value, such as the property \textit{native languange}. Both of these properties have values that are other Wikidata entities. Additionally, there are other types of properties, such as \textit{population}, which has a static numerical value, and \textit{Freebase ID}, which provides a string used to identify the entity in the Freebase database.

While the previous examples focus on information with outgoing links, we can also consider the opposite perspective by examining incoming links. This approach does not directly show the possessions of an entity but rather indicates its popularity by showing how often it is mentioned elsewhere. This is illustrated by the Romani people being mentioned in \textit{Time of the Gypsies} and the Minangkabau being associated with Sheikh Tahir Jalaluddin.

The example in \autoref{fig:intro-wikidata} provides a clear picture of how entities in Wikidata can have different kinds and amounts of information, which may lead to knowledge imbalance. If this issue is left unaddressed, it can be problematic for anyone utilizing open KGs as a data source. Data users may draw invalid inferences and conclusions based on incomplete or imbalanced data, such as the Minangkabau and Ambonese are less important than the Romani. Moreover, if contributors to open KGs cannot identify which entities or classes are lacking information, efforts such as editathons may not be effective, potentially widening the gap between information-rich and information-poor entities. For example, contributors might prioritize enriching the Romani people's data while overlooking gaps in the Minangkabau entry, leaving the \textit{flag} property in the Minangkabau empty despite the well-documented existence of the Marawa flag of Minangkabau.

% Existing approaches often focus on ...., lacking a way of quantifying the amount of information contained in KGs. Our study addresses this gap by proposing a formal model to define the knowledge wealth in the RDF knowledge graph. Additionally, we construct an analytical framework using statistical measures and visualization to give insights about the wealth of a class, the inequality between classes, and imbalance measure of wealth within a class. To evaluate this framework, we conducted several use cases on various classes in Wikidata.

% Our contributions are:
% \begin{enumerate}
%     \item We introduce the 3 notions of quantifying knowledge wealth for knowledge graphs, and show how they can be used to further characterize knowledge wealth in knowledge graphs.
%     \item We implemented the formal and insight model using Python programming language and made it accessible.
%     \item We perform a case study on Wikidata classes, showing how biases can be identified in Wikidata, how different definition of wealth impacts the imbalance level of a class, and how the composisition of wealth based on is.
% \end{enumerate}

Existing approaches often focus on (...), lacking a way of quantifying the amount of information contained in KGs. Our study addresses this gap by proposing a formal model to define the knowledge wealth in the RDF knowledge graph. Specifically, we focus on three key contributions: (i) introducing three notions of quantifying knowledge wealth for knowledge graphs and demonstrating how they can be used to further characterize knowledge wealth; (ii) implementing the formal and insight model using Python and making it accessible for broader use; and (iii) conducting a case study on Wikidata classes, illustrating how biases can be identified, how different definitions of wealth impact the imbalance level of a class, and how the composition of wealth varies between classes.

%% can be formal background and literature studies
\section{Related Work}

\paragraph{Data Completeness Profiling} Wisesa et al. (2019) presented ProWD, a framework and web application tool for profiling the completeness of Wikidata. It is used to provide insight on degree of attribute completeness of a class in Wikidata. The visualization provided in the we dashboard is equipped with single, compare, or multidimensional view to help in analyzing the facet at entity or class level.

\paragraph{Imbalance and Gap in Wikidata}
- Refo: Gini index

- Nio: gap property
Ramadizsa1 et al. (2023) introduced the concept of gap properties that helps to characterize class-level knowledge gaps within knowledge graphs. The framework adapts association rule mining to determine ...

%% can be formalization
\section{Knowledge Wealth Framework}

% - RDF model, triples
% - definition of wealth berdasarkan property (kardinalitas, tipe properti, arah properti)
% - insight model

In this section, we define the formal framework for measuring knowledge wealth in knowledge graphs (KGs). We start by introducing the underlying data model based on RDF structure and how entities are represented through triples. Then, we formalize different notions of knowledge wealth using property-based metrics. Lastly, we present an insight model that supports both quantitative analysis and qualitative interpretation of wealth distributions across entity classes.

\subsection{Wealth Formal Model} \label{wealth-formal-model}
Knowledge graphs follow Resource Description Framework (RDF)~\cite{w3crdf} as a means of data organization. Without loss of generality of how the form of the URIs is, data is stored in the form of triple \((s, p, o)\); a combination of a subject \(s\), a predicate \(p\), and an object \(o\) which can be visualized as nodes and directed-arc diagrams. For example, the statement "William Shakespeare's notable work is Romeo and Juliet" in human-readable URIs is mapped to the triple (\textit{WilliamShakespeare}, \textit{notableWork}, \textit{RomeoAndJuliet}). Likewise, the statement in Wikidata, which uses ID-based URIs, is mapped to (\textit{Q692}, \textit{P800}, \textit{Q83186}).

There are 2 kinds of nodes: IRIs and literals. A triple is in the form of \((s, p, o) \in I \times I \times (I \cup L) \) where \(I\) is the node with type IRI and \(L\) is the node with type literal. A knowledge graph is a finite set of triples \((s, p, o)\).

% \subsubsection{Class}
% In this study, we re-use the class model defined by Ramadizsa et al. (2023). A class is a group of entities that are the subject of the study. \textit{Human}, \textit{film}, and \textit{taxon} are some examples of class. In general, entity \(s\) is an instance of class \(C\) is expressed by the triple (\(s\), \textit{instanceOf}, \(C\)) or (\(s\), \textit{type}, \(C\)). We can get a more narrow class inside the defined class by specifying additional conditions, each consisting of a particular property and value associated with it. Example of such conditions for human class is \textit{gender} with associated value \textit{male}, while example for a country would be \textit{continent} with value \textit{Asia}. For instance, the class of human with gender male that lived during English Renaissance is queried using (\(?s\), \{(\(?s\), \textit{instanceOf}, \textit{human}), (\(?s\), \textit{gender}, \textit{male}), (\(?s\), \textit{timePeriod}, \textit{EnglishRenaissance})\}).

\subsubsection{Entity-Level Wealth: Knowledge Wealth Type and Definition}

Let \(e \in I\) be an entity in a knowledge graph \(G\). We quantify the wealth of entity \(e\) in \(G\) as the amount of information about \(e\) available in \(G\). Thus, we define the knowledge wealth of an entity as the number of properties associated/linked to it. For example, the wealth of \textit{William Shakespeare} (\textit{Q692}) in Wikidata counts all triples describing \textit{Q692} in Wikidata, including those detailing his family, occupation, image, and so on.

As motivated in \Cref{introduction}, there are three key dimensions to consider when alculating the knowledge wealth of an entity: (1) wealth based on the (non-)uniqueness or multiplicity of individual property; (2) wealth based on type of property; and (3) wealth based on the direction of the link. We denote the knowledge wealth of an entity \(e\) in graph \(G\) using the general form:
\[
    W_{mult,[prop],[dir]}(e,G)
\]
where:
\[
    mult \in  \{bag, set\}
\]
\[
    prop \subseteq  \{obj, ltr, id\}
\]
\[
    dir \subseteq  \{out, in\}
\]
This formalization provides a flexible and expressive way to define and compute knowledge wealth under different configurations.

\begin{figure}[!h]
    \centering
    \begin{subfigure}[b]{0.3\textwidth}
        \centering
        \includegraphics[scale=.3]{Wealth Type 1}
        \caption{Illustration of bag of properties and set of properties} \label{fig:wealth-type1}
    \end{subfigure}
    \hfill
    \begin{subfigure}[b]{0.3\textwidth}
        \centering
        \includegraphics[scale=.3]{Wealth Type 2}
        \caption{Illustration of 3 types of property: object, literal, ID} \label{fig:wealth-type2}
    \end{subfigure}
    \hfill
    \begin{subfigure}[b]{0.3\textwidth}
        \centering
        \includegraphics[scale=.3]{Wealth Type 3}
        \caption{Illustration of incoming link vs. outgoing link} \label{fig:wealth-type3}
    \end{subfigure}
    \caption{Three simple graphs} \label{fig:three graphs}
\end{figure}

\paragraph{Wealth based on the (non-)uniqueness of individual properties.}
The first aspect of knowledge wealth concerns how we regard the multiplicity or the (non-)uniqueness or the properties. Specifically, if an entity \(e\) has a property \(p_1\) that links it to a finite set of \(n\) distinct objects \(o_1\), \(o_2\), ..., \(o_n\), we may choose whether to account for the total number of objects associated with it (which is \(n\)), or to consider only the existence of \(p_1\). We refer to the former as the bag of properties, and the latter as the set of properties approach.

Under the bag of properties, an entity's knowledge wealth is defined as the cardinality of the set of triples in which it appears (either in the subject or object position, depending on the assumed link direction). For example, assuming an outgoing link direction, the triples \((e, p_1, o_1)\) and \((e, p_1, o_2)\) account for a wealth of 1 each, thus both have a total of 2. Formally, the knowledge wealth of an entity \(e\) under the bag of properties approach when no filtering is applied to the property type and using outgoing link direction, is denoted as \(W_{bag, *, [out]}(e,G)\) and is defined as:
% Similarly, if the incoming link direction is used, the triples \((s_1, p_1, e)\) and \((s_2, p_1, e)\) each contribute a wealth of 1, again totaling a wealth of 2. Formally, the knowledge wealth of an entity \(e\) under the bag of properties approach when no filtering is applied to the property type, is denoted as \(W_{bag, *, out}(e,G)\) when using outgoing link, and \(W_{bag, *, in}(e,G)\) when using incoming link, and is defined as:
\[
    W_{bag, *, [out]}(e,G) = |\{(p, o) | (e, p, o) \in G\}|
\]
% \[
%     W_{bag, *, in}(e,G) = |\{(s, p) | (s, p, e) \in G\}|
% \]

In the set of properties, knowledge wealth is measured by counting the number of distinct properties that describe the entity. By this way, we are capturing the variety of information about an entity. In contrast to bag of properties, in set of properties (with an assumption of outgoing link direction), \((e, p_1, o_1)\) and \((e, p_1, o_2)\) would be regarded as the "same" information because of the identical property \(p_1\), thus they only account for a total wealth of 1. Formally, the knowledge wealth of an entity \(e\) under the set of properties approach when no filtering is applied to the property type and using outgoing link direction, is denoted as \(W_{set, *, [out]}(e,G)\) and is defined as:
% Likewise, when using the incoming link direction, \((s_1, p_1, e)\) and \((s_2, p_1, e)\) are also treated as identical and collectively count as a single contribution to the entity's wealth. Formally, the knowledge wealth of an entity \(e\) under the set of properties approach when no filtering is applied to the property type, is denoted as \(W_{set, *, out}(e,G)\) when using outgoing link, and \(W_{set, *, in}(e,G)\) when using incoming link, and is defined as:
\[
    W_{set, *, [out]}(e,G) = |\{p | (e, p, o) \in G\}|
\]
% \[
%     W_{set, *, in}(e,G) = |\{(p) | (s, p, e) \in G\}|
% \]

% The first measure of the knowledge wealth of \(s\) is bag of properties---the cardinality of set of all triples that has \(s\) in their subject position. In this definition, the triples \((s, p_1, o_1)\) and \((s, p_1, o_2)\) account for a wealth of 1 each, thus both have a total of 2.

% Let \(N_{bag}(s,G)\) be a set that comprises all pair of predicate/property and object \((p,o)\) that is connected to \(s\). Then \(W_{bag}(s, G)\) is the cardinality of \(N_{bag}(s,G)\).
% \[
%     N_{bag}(s,G) = \{(p, o) | (s, p, o) \in G\}
% \]
% \[
%     W_{bag}(s,G) = |N_{bag}(s)|
% \]

% Another way of measuring the wealth is by counting the number of distinct properties describing the entity, or set of properties. By this way, we are capturing the variety of information about an entity. In contrast to bag of properties, in set of properties \((s, p_1, o_1)\) and \((s, p_1, o_2)\) would be regarded as the "same" information because of the identical property \(p_1\), thus they only account for a total wealth of 1. Let \(N_{set}(s,G)\) be a set that comprises all predicate/property \(p\) that is connected to \(s\). Then \(W_{set}(s, G)\) is the cardinality of \(N_{set}(s,G)\).
% \[
%     N_{set}(s, G) = \{p | \exists o, (s, p, o) \in G\}
% \]
% \[
%     W_{set}(s, G) = |N_{set}(s,G)|
% \]

% By the above definition, the wealth of entity \(s\) in \autoref{fig:wealth-type1} is 3 and 2, using bag of properties and set of properties respectively.

\paragraph{Wealth based on type of property.}
There are two main categories of properties in RDF: object properties, which link IRIs (on subject position) to IRIs (on object position), and datatype properties, which link IRIs (on subject position) to data literals (on object position)~\cite{w3crdf}. The idea of knowledge wealth based on property type is inspired by how human wealth is assessed across different types of possessions (e.g., real estate, vehicles, financial assets). Similarly, in knowledge graphs, property types can be seen as categories of informational assets that contribute to an entity's overall wealth.

This formulation allows us to evaluate not just how much information an entity has, but what kind of information it possesses. For instance, one entity may be rich in object links (e.g., affiliations or relationships to other entities), while another may be rich in literal facts (e.g., birth date, height). Segmenting wealth by property type thus provides a more nuanced view of an entity's informational profile.

Since each property \(p \in I\), we define the set of object properties as \(I_{objProp}\). Therefore, the knowledge wealth of an entity \(e\), based solely on object properties and using the bag of properties approach (assuming outgoing link direction) is defined as:
\[
    W_{bag,[obj],[out]}(e,G) = |\{(p,o) | ((e, p, o) \in G) \cap (p \in I_{objProp})\}|
\]

The handling for datatype properties is special. Among literal values, we see that there are two categories: pure literals, which are values that behave like constants or factual attributes (e.g., date of birth, height, population), and (external) IDs, which are string-based values that are used as identifier for the entity in external systems or databases (e.g., Freebase ID, IMDb ID). Based on this distinction, we define two corresponding subsets of datatype properties. The first one is \(I_{ltrProp}\) for the set of pure literals properties, and \(I_{idProp}\) for the set of IDs properties. Then, the knowledge wealth of an entity \(e\) under the bag of properties approach, using outgoing links, can be separately defined for each category as:
\[
    W_{bag,[ltr],[out]}(e,G) = |\{(p,o) | ((e, p, o) \in G) \cap (p \in I_{ltrProp})\}|
\]
\[
    W_{bag,[id],[out]}(e,G) = |\{(p,o) | ((e, p, o) \in G) \cap (p \in I_{idProp})\}|
\]

The sets \(I_{objProp}\), \(I_{ltrProp}\), and \(I_{idProp}\) are all mutually exclusive. We define their union as \(I_{Prop}\), the set of IRIs that function as properties (i.e., those used in the predicate position of RDF triples but not act as entity subjects or objects). Formally, this set is defined as:
\[
    I_{Prop} = (I_{objProp} \cup I_{ltrProp} \cup I_{idProp})
\]

When measuring the knowledge wealth of an entity \(e\), we may choose to filter by a specific property type or take a combination of them. For instance, in the previously defined measure \(W_{bag, *, [out]}(e,G)\), the assumption is that all property types are included (i.e., \(p \in I_{Prop}\)), while \(W_{bag, [obj, ltr], [out]}(e,G)\) includes only object and literal properties (i.e., \(p \in (I_{objProp} \cup I_{ltrProp})\))

% Object properties are other entities besides \(s\) that is connected with \(s\) through a property \(p\). Wealth of \(s\) using bag of properties with only considering the object properties is defined as:
% \[
%     W_{bag, object}(s, G) = |\{(p,o) | ((s, p, o) \in G) \cap (o \in I)\}|
% \]
% Literal properties are non-object properties that is connected with \(s\) through a property \(p\). Wealth of \(s\) using bag of properties with only considering the literal properties is defined as:
% \[
%     W_{bag, literal}(s, G) = |\{(p,o) | ((s, p, o) \in G) \cap (o \in L)\}|
% \]

% An external ID is a special type of string that is used to represent an entity in an external source. In Wikidata, an ID is identifiable by property type \textit{wikibase:ExternalId}. Just like any other property, an ID is connected with \(s\) through a property \(p\). Let  \(C_{ID,G}\) be a set comprising ID property in graph \(G\). Wealth of \(s\) using bag of properties with only considering the ID properties is defined as:
% \[
%     W_{bag, ID}(s, G) = |\{(p,o) | ((s, p, o) \in G) \cap (o \in L) \cap (o \in C_{ID,G})\}|
% \]

\paragraph{Wealth based on the direction of the link.}
In the previous definitions of knowledge wealth, we assumed that links originate from the entity \(e\); that is, \(e\) appears as the subject in the set of triples in graph \(G\). However, wealth can also be defined from the opposite perspective, where links point toward the entity. In this incoming link formulation, the properties contributing to the wealth of an entity \(e\) are derived from triples in which \(e\) appears as the object.

Under the bag of properties approach, in contrast to the outgoing link formulation where we count the pairs \(p, o\) from triples \(e, p, o\), in the incoming link case we count the pairs \(s, p\) from triples \(s, p, e\). In this definition, the triples \((s_1, p_1, e)\) and \((s_2, p_1, e)\) each contribute a wealth of 1 since the pairs \(s_1, p_1\) and \(s_2, p_1\) are treated as distinct. This results in a total wealth of 2 for the entity \(e\). Formally, the knowledge wealth of an entity \(e\) in this case is denoted as \(W_{bag, *, [in]}(e,G)\), and is defined as:

\[
    W_{bag, *, [in]}(e,G) = |\{(s, p) | (s, p, e) \in G\}|
\]

Likewise, under the set of properties approach and incoming link direction, the triples \((s_1, p_1, e)\) and \((s_2, p_1, e)\) are treated as identical since they share the same property \(p_1\), and collectively count as a single contribution to the entity's wealth. The corresponding formal definition for this case is denoted as \(W_{set, *, [in]}(e,G)\), and is defined as:

\[
    W_{set, *, [in]}(e,G) = |\{(p) | (s, p, e) \in G\}|
\]

It is important to note that, for the incoming link formulation, we always assume that \(p \in I_{objProp}\). This is because the object position in the triples is taken by the entity \(e\), and \(e \in I\) (\(e\) is not a literal/ID).

% In outgoing link type of wealth, the properties that are used in the wealth calculation of an entity \(s\) are those obtained from link with outwards direction from that particular entity \(s\); that is where \(s\) appears to be the subject in the set of triples in graph \(G\). All types of wealth defined before use the notion of outgoing link.

% In incoming link type of wealth, the properties that are used in the wealth calculation of an entity \(s\) are those obtained from link with inwards direction to that particular entity \(s\); that is where \(s\) appears to be the object in the set of triples in graph \(G\). To illustrate, let \(N_{bag}(s)\) be a set that comprises all pair of object and predicate/property \((o,p)\) that is connected to \(s\) in incoming direction to \(s\) i.e., \(N_{bag}(s)\) = \(\{(o, p) | (o, p, s) \in G\}\). Then the wealth of \(s\) using bag of properties and the view of incoming link is notated as \(W_{bag, incoming}(s, G)\), and equal to the cardinality of \(N_{bag}(s)\).
% \[
%     N_{bag, incoming}(s, G) = \{(o,p) | (o, p, s) \in G\}
% \]
% \[
%     W_{bag, incoming}(s, G) = |N_{bag, incoming}(s, G)|
% \]

% Looking at in \autoref{fig:wealth-type3}, the wealth of entity \(s\) is 1 using outgoing link, which is from the triple \((s, p_1, o1)\). Its wealth is also 1 and using incoming link, which comes from the triple \((o2, p_2, s)\).

% Each definition above can be used simultaneously. For example, the wealth of entity \(s\) using set of properties, calculating object and data but not ID properties, and using the direction of outgoing link is denoted by \(W_{set, outgoing, (object \cup data) \cap ID^\complement}(s, G) = |N_{set, outgoing, (object \cup data) \cap ID^\complement}(s, G)|\) with \(N_{set, outgoing, (object \cup data) \cap ID^\complement}(s, G) = \{p | \exists o, (s, p, o) \in G, \cap (o \in ((I \cup L) \cap C_{ID,G}^\complement))\}\).

\begin{center}
    \scriptsize
    \makebox[\linewidth]{
    \begin{threeparttable}
    \captionsetup{font=small}
    \caption{Wealth Formal Definitions}
    \label{tab:wealth-formal-definition}
    \begin{tabular}{|c | c c c c c c |} 
    
    \toprule
        No & Formal Notation & Multiplicity & \CellWithForceBreak{Property \\ Type} & \CellWithForceBreak{Link \\ Direction} & Formal Definition & Notes \\ [0.5ex] 
    \midrule
        1 & $W_{bag,*,[out]}(e,G)$ & Bag & * & Out & $|\{(p, o) | (e, p, o) \in G\}|$ & \\
        2 & $W_{bag,[obj],[out]}(e,G)$ & Bag & Object & Out & $|\{(p,o) | ((e, p, o) \in G) \cap (p \in I_{objProp})\}|$ & \\
        3 & $W_{bag,[ltr],[out]}(e,G)$ & Bag & Literal & Out & $|\{(p,o) | ((e, p, o) \in G) \cap (p \in I_{ltrProp})\}|$ & \\
        4 & $W_{bag,[id],[out]}(e,G)$ & Bag & ID & Out & $|\{(p,o) | ((e, p, o) \in G) \cap (p \in I_{idProp})\}|$ & \\
        5 & $W_{bag,[obj, ltr],[out]}(e,G)$ & Bag & Object, literal & Out & $|\{(p,o) | ((e, p, o) \in G) \cap (p \in (I_{objProp} \cup I_{ltrProp}))\}|$ & \\
        6 & $W_{bag,[obj, id],[out]}(e,G)$ & Bag & Object, ID & Out & $|\{(p,o) | ((e, p, o) \in G) \cap (p \in (I_{objProp} \cup I_{idProp}))\}|$ & \\
        7 & $W_{bag,[ltr, id],[out]}(e,G)$ & Bag & Literal, ID & Out & $|\{(p,o) | ((e, p, o) \in G) \cap (p \in (I_{ltrProp} \cup I_{idProp}))\}|$ & \\

        8 & $W_{bag,*,[in]}(e,G)$ & Bag & * & In & $|\{(s, p) | (s, p, e) \in G\}|$ & \\
        9 & $W_{bag,[obj],[in]}(e,G)$ & Bag & Object & In & $|\{(s,p) | ((s, p, e) \in G) \cap (p \in I_{objProp})\}|$ & \\
        10 & $W_{bag,[ltr],[in]}(e,G)$ & Bag & Literal & In & $|\emptyset| = 0$ & \CellWithForceBreak{$p \in I_{objProp}$ \\ because $e \in I$} \\
        11 & $W_{bag,[id],[in]}(e,G)$ & Bag & ID & In & $|\emptyset| = 0$ & \CellWithForceBreak{$p \in I_{objProp}$ \\ because $e \in I$} \\
        12 & $W_{bag,[obj, ltr],[in]}(e,G)$ & Bag & Object, literal & In & $|\{(s,p) | ((s, p, e) \in G) \cap (p \in I_{objProp})\}|$ & \CellWithForceBreak{$p \in I_{objProp}$ \\ because $e \in I$} \\
        13 & $W_{bag,[obj, id],[in]}(e,G)$ & Bag & Object, ID & In & $|\{(s,p) | ((s, p, e) \in G) \cap (p \in I_{objProp})\}|$ & \CellWithForceBreak{$p \in I_{objProp}$ \\ because $e \in I$} \\
        14 & $W_{bag,[ltr, id],[in]}(e,G)$ & Bag & Literal, ID & In & $|\emptyset| = 0$ & \CellWithForceBreak{$p \in I_{objProp}$ \\ because $e \in I$} \\

        15 & $W_{bag,*,*}(e,G)$ & Bag & * & * & $|\{(p, o) | (e, p, o) \in G\}| + |\{(s, p) | (s, p, e) \in G\}|$ & \CellWithForceBreak{Aggregation of \\ \#1 and \#8} \\
        16 & $W_{bag,[obj],*}(e,G)$ & Bag & Object & * &  \CellWithForceBreak{$|\{(p,o) | ((e, p, o) \in G) \cap (p \in I_{objProp})\}|$ \\ $ + |\{(s,p) | ((s, p, e) \in G) \cap (p \in I_{objProp})\}|$} & \CellWithForceBreak{Aggregation of \\ \#2 and \#9} \\
        17 & $W_{bag,[ltr],*}(e,G)$ & Bag & Literal & * & $|\{(p,o) | ((e, p, o) \in G) \cap (p \in I_{ltrProp})\}|$ & \CellWithForceBreak{Aggregation of \\ \#3 and \#10} \\
        18 & $W_{bag,[id],*}(e,G)$ & Bag & ID & * & $|\{(p,o) | ((e, p, o) \in G) \cap (p \in I_{idProp})\}|$ & \CellWithForceBreak{Aggregation of \\ \#4 and \#11} \\
        19 & $W_{bag,[obj, ltr],*}(e,G)$ & Bag & Object, literal & * & \CellWithForceBreak{$|\{(p,o) | ((e, p, o) \in G) \cap (p \in (I_{objProp} \cup I_{ltrProp}))\}|$  \\ $ + |\{(s,p) | ((s, p, e) \in G) \cap (p \in I_{objProp})\}|$} & \CellWithForceBreak{Aggregation of \\ \#5 and \#12} \\
        20 & $W_{bag,[obj, id],*}(e,G)$ & Bag & Object, ID & * & \CellWithForceBreak{$|\{(p,o) | ((e, p, o) \in G) \cap (p \in (I_{objProp} \cup I_{idProp}))\}|$ \\ $ + |\{(s,p) | ((s, p, e) \in G) \cap (p \in I_{objProp})\}|$} & \CellWithForceBreak{Aggregation of \\ \#6 and \#13} \\
        21 & $W_{bag,[ltr, id],*}(e,G)$ & Bag & Literal, ID & * & $|\{(p,o) | ((e, p, o) \in G) \cap (p \in (I_{ltrProp} \cup I_{idProp}))\}|$ & \CellWithForceBreak{Aggregation of \\ \#7 and \#14} \\

        22 & $W_{set,*,[out]}(e,G)$ & Set & * & Out & $|\{p | (e, p, o) \in G\}|$ & \\
        23 & $W_{set,[obj],[out]}(e,G)$ & Set & Object & Out & $|\{p | ((e, p, o) \in G) \cap (p \in I_{objProp})\}|$ & \\
        24 & $W_{set,[ltr],[out]}(e,G)$ & Set & Literal & Out & $|\{p | ((e, p, o) \in G) \cap (p \in I_{ltrProp})\}|$ & \\
        25 & $W_{set,[id],[out]}(e,G)$ & Set & ID & Out & $|\{p | ((e, p, o) \in G) \cap (p \in I_{idProp})\}|$ & \\
        26 & $W_{set,[obj, ltr],[out]}(e,G)$ & Set & Object, literal & Out & $|\{p | ((e, p, o) \in G) \cap (p \in (I_{objProp} \cup I_{ltrProp}))\}|$ & \\
        27 & $W_{set,[obj, id],[out]}(e,G)$ & Set & Object, ID & Out & $|\{p | ((e, p, o) \in G) \cap (p \in (I_{objProp} \cup I_{idProp}))\}|$ & \\
        28 & $W_{set,[ltr, id],[out]}(e,G)$ & Set & Literal, ID & Out & $|\{p | ((e, p, o) \in G) \cap (p \in (I_{ltrProp} \cup I_{idProp}))\}|$ & \\
        29 & $W_{set,*,[in]}(e,G)$ & Set & * & In & $|\{p | (s, p, e) \in G\}|$ & \\
        30 & $W_{set,[obj],[in]}(e,G)$ & Set & Object & In & $|\{p | ((s, p, e) \in G) \cap (p \in I_{objProp})\}|$ & \\
        31 & $W_{set,[ltr],[in]}(e,G)$ & Set & Literal & In & $|\emptyset| = 0$ & \CellWithForceBreak{$p \in I_{objProp}$ \\ because $e \in I$} \\
        32 & $W_{set,[id],[in]}(e,G)$ & Set & ID & In & $|\emptyset| = 0$ & \CellWithForceBreak{$p \in I_{objProp}$ \\ because $e \in I$} \\
        33 & $W_{set,[obj, ltr],[in]}(e,G)$ & Set & Object, literal & In & $|\{p | ((s, p, e) \in G) \cap (p \in I_{objProp})\}|$ & \CellWithForceBreak{$p \in I_{objProp}$ \\ because $e \in I$} \\
        34 & $W_{set,[obj, id],[in]}(e,G)$ & Set & Object, ID & In & $|\{p | ((s, p, e) \in G) \cap (p \in I_{objProp})\}|$ & \CellWithForceBreak{$p \in I_{objProp}$ \\ because $e \in I$} \\
        35 & $W_{set,[ltr, id],[in]}(e,G)$ & Set & Literal, ID & In & $|\emptyset| = 0$ & \CellWithForceBreak{$p \in I_{objProp}$ \\ because $e \in I$} \\
        36 & $W_{set,*,*}(e,G)$ & Set & * & * & $|\{p | (e, p, o) \in G\}| + |\{p | (s, p, e) \in G\}|$ & \CellWithForceBreak{Aggregation of \\ \#22 and \#29}\\
        37 & $W_{set,[obj],[out]}(e,G)$ & Set & Object & Out & \CellWithForceBreak{$|\{p | ((e, p, o) \in G) \cap (p \in I_{objProp})\}|$ \\ $+ |\{p | ((s, p, e) \in G) \cap (p \in I_{objProp})\}|$} & \CellWithForceBreak{Aggregation of \\ \#23 and \#30}\\
        38 & $W_{set,[ltr],[out]}(e,G)$ & Set & Literal & Out & $|\{p | ((e, p, o) \in G) \cap (p \in I_{ltrProp})\}|$ & \CellWithForceBreak{Aggregation of \\ \#24 and \#31}\\
        39 & $W_{set,[id],[out]}(e,G)$ & Set & ID & Out & $|\{p | ((e, p, o) \in G) \cap (p \in I_{idProp})\}|$ & \CellWithForceBreak{Aggregation of \\ \#25 and \#32}\\
        40 & $W_{set,[obj, ltr],[out]}(e,G)$ & Set & Object, literal & Out & \CellWithForceBreak{$|\{p | ((e, p, o) \in G) \cap (p \in (I_{objProp} \cup I_{ltrProp}))\}|$ \\ $+ |\{p | ((s, p, e) \in G) \cap (p \in I_{objProp})\}|$} & \CellWithForceBreak{Aggregation of \\ \#26 and \#33}\\
        41 & $W_{set,[obj, id],[out]}(e,G)$ & Set & Object, ID & Out & \CellWithForceBreak{$|\{p | ((e, p, o) \in G) \cap (p \in (I_{objProp} \cup I_{idProp}))\}|$ \\ $+ |\{p | ((s, p, e) \in G) \cap (p \in I_{objProp})\}|$} & \CellWithForceBreak{Aggregation of \\ \#27 and \#34}\\
        42 & $W_{set,[ltr, id],[out]}(e,G)$ & Set & Literal, ID & Out & $|\{p | ((e, p, o) \in G) \cap (p \in (I_{ltrProp} \cup I_{idProp}))\}|$ & \CellWithForceBreak{Aggregation of \\ \#28 and \#35}\\

    \bottomrule
    \end{tabular}
    \begin{tablenotes}
        \scriptsize
        \item{This table summarizes the 30 formal definitions of knowledge wealth used in this study. Each row represents a variant of wealth based on three key dimensions: (1) multiplicity: bag or set; (2) property type: object, literal, ID, or any of their combinations; and (3) link direction: in, out, or any of their combinations. In the notation used in the table, \(G\) denotes the RDF knowledge graph, \(e\) is an entity (instance) in that graph. Each triple in the graph is composed of a subject \(s\), a property \(p\), and an object \(o\). \(I\) the set of all IRIs in the graph, \(I_objProp\) is the set of IRIs used as object properties, \(I_ltrProp\) is the set of IRIs used as pure literal properties, and \(I_idProp\) is the set of IRIs used as identifier properties. \(I_{Prop} = (I_{objProp} \cup I_{ltrProp} \cup I_{idProp})\) is the set of all valid property IRIs.}
    \end{tablenotes}
    \end{threeparttable}
    }
\end{center}

\subsubsection{Class-Level Wealth}
In this study, we re-use the class model defined by Ramadizsa et al.~\cite{RamadizsaDNR23}. A class is a group of entities that are the subject of the study. \textit{Human}, \textit{film}, and \textit{taxon} are some examples of class. In general, entity \(s\) is an instance of class \(C\) is expressed by the triple (\(s\), \textit{instanceOf}, \(C\)) or (\(s\), \textit{type}, \(C\)). We can get a more narrow class inside the defined class by specifying additional conditions, each consisting of a particular property and value associated with it. Example of such conditions for human class is \textit{gender} with associated value \textit{male}, while example for a country would be \textit{continent} with value \textit{Asia}. For instance, the class of human with gender male that lived during English Renaissance is queried using (\(?s\), \{(\(?s\), \textit{instanceOf}, \textit{human}), (\(?s\), \textit{gender}, \textit{male}), (\(?s\), \textit{timePeriod}, \textit{EnglishRenaissance})\}).

Let \(C\) be a class that consists of \(m\) distinct entities \(s_1\), \(s_2\), ... \(s_m\) in graph \(G\). We define \(T_C\) a multiset consists of the wealth of each entity of \(C\), i.e., 
\[
    T_{C,G} = \{W(s_1), G, W(s_2, G), W(s_3, G), ..., W(s_m, G)\}
\] 
We denote the overall wealth of class \(C\) as \(\mathcal{W}(C)\). It can be quantified using its constituent entities and be expressed as a function of \(T_C\), denoted as \(f(T_C)\), where \(f\) may represent statistical summaries such as the cardinality (i.e., entity count), sum, mean, median, mode, or percentile of its entities's wealth.
\[
    \mathcal{W}(C,G) = f(T_{C,G})
\]
As an example, consider a class \(C\) consisting of of 4 entities  \(s_1\), \(s_2\), \(s_3\), and \(s_4\) from \autoref{fig:wealth-weighted}. If we use cardinality as the function \(f\), then \(\mathcal{W}(C) = 4\), corresponding to the number of entities it contains.

It is important to note that not all statistical functions are suitable for this purpose. The previously mentioned functions provide interpretable measures of central tendency or overall volume, thus they are considered valid to be used as \(f\) to represent the level of wealth of a class. On the contrary, inequality metrics such as the Gini coefficient or standard deviation are designed to capture distributional imbalance rather than the actual level of wealth, and therefore are not appropriate as definitions of \(\mathcal{W}(C)\), which aims to describe how rich a class is, and not how unequal the wealth distribution is within it.
\subsection{Insight Model} \label{insight-model}

\paragraph{Exploratory Data Analysis (EDA): Descriptive Statistics Measures.} Descriptive statistics is concerned with the description and summarization of data. It is a summary of a dataset that helps to describe features of data quantitatively (Ross, 2019). To have a general view of wealth distribution of a class, we use the following measures:
\begin{itemize}
  \item measures of central tendency: mean, median, mode
  \item measures of frequency: count, cumulative frequency/percentage
  \item measures of position: quartile, percentile
  \item measures of dispersion: minimum, maximum, range, interquartile range, standard deviation, coefficient of variation, kurtosis
  \item measures of symmetry: skewness
\end{itemize}


\paragraph{Gini Coefficient.}
Gini coefficient is a metric used to measure the economic wealth gaps between countries. A study by Akbar (2020) utilized the Gini coefficient to measure the level of knowledge imbalance in knowledge graphs, particularly Wikidata classes. To calculate the imbalance level of a Wikidata class using Gini coefficient, the researcher started by calculating the number of properties of each entity of that particular class and storing them in an array. The array will then be sorted in descending order, from the smallest to the largest i.e., \(y_{i} \geq y_{i+1} \forall i \in \{1, 2, ..., n\}\). The Gini coefficient will be calculated from the sorted array using the Gini coefficient formula below~\cite{FosterS1997}.

\[G = 1 - \frac{1}{n^{2}\mu} \sum_{i=1}^{n} \sum_{j=1}^{n} Min(y_{i}, y_{j})\]
\[G = 1 + \frac{1}{n} - \frac{1}{n^{2}\mu} - (y_{1} + 2y_{2} + ... + ny_{n})\]

In economics context, \(n\) is the size of population of a country, $\mu$ is the average income, and \(y\) is an array containing data of each country's income. However, in the context of knowledge graph, \(n\) is the number of entities in the class, $\mu$ is the average knowledge wealth of the entities, and \(y\) is an array containing data of each entity's wealth, sorted in descending order.

For example, let's say we have a class that consists of 10 entities. After counting the number of properties of each entities (using the notion of bag of properties for wealth) and sorting them in descending order, we will have an array of \(y = [10,8,8,7,4,2,2,1,1,1]\). The length of the array is \(n = 10\) and the average wealth is \(\mu = 4.4\). Then, apply the Gini coefficient formula and we get \(G = 1 + \frac{1}{10} - \frac{2}{10^{2}\times4.4}(1\times10 + 2\times8 + ... + 10\times1) = 0.414\)

For another another example, let's say we have another array of 10 entities \(z = [10, 9, 9, 9, 9, 9, 9, 9, 9, 5]\). By applying the same formula to z, we get a Gini coefficient value of \(0.052\).

The Gini coefficient has a value between \(0\) and \(1\). The higher the coefficient value, the greater the imbalance level. The value of 0 is achieved when all observed entities have the same amount of wealth. The value of 1 occurs when all income is owned solely by one entity and this phenomenon expresses full inequality.


\paragraph{Lorenz Curve.} Lorenz curve is a graphical representation of wealth distribution and its inequality~\cite{Kakwani1977}. It shows how the wealth is cumulatively distributed, with data points sorted in ascending order from the poorest to the richest. In Lorenz curve, the horizontal axis represents the fraction of the population, and the vertical axis represents the cumulative wealth. Therefore, if the point (\textit{x}, \textit{y}) = (30, 15) lies on the curve, then we can interpret that the bottom 30\% of the population account for 15\% of the total wealth in that population. The Lorenz curve is usually drawn along with a straight diagonal line with a slope of 1. This straight line represents perfect equality in wealth distribution, i.e., each individual in the observed population has equal wealth. The Lorenz curve itself is drawn below the straight line. The ratio of the area between the Lorenz curve and the straight line of perfect equality to the triangular area below the straight line, is the Gini coefficient.
\subsection{Sample Application of Formal and Insight Model}

We provide small case in \autoref{fig:wealth-weighted} to illustrate how both models can be applied in quantifying knowledge wealth. In this example, we will focus on the notion using bag of properties with outgoing direction of link to quantify the wealth.

\begin{figure}[!h]
    \centering
    \includegraphics[scale=.3]{Wealth Weighted}
    \caption{Sample knowledge graph \(G\) that contains class \(C\) with 4 entities} \label{fig:wealth-weighted}
\end{figure}

A graph \(G\) has a class \(C\), which consists of 4 entities \(s_1\), \(s_2\), \(s_3\), and \(s_4\). Using bag of properties and outgoing link direction, the walth of entity \(s_1\), \(s_2\), \(s_3\), and \(s_4\) are 3, 3, 6, and 2 respectively. \autoref{tab:sample statistical summary} shows some statistical summary that describe the wealth of class \(C\).

\begin{center}
    \small
    \begin{threeparttable}
    \caption{Statistical Summary of Wealth of Class \(C\)}
    \label{tab:sample statistical summary}
    \begin{tabular}{c | c c c c c c c} 
    
    \toprule
        Measure & Entity Count & Mean & Median & Mode & Minimum & Maximum & Gini \\ [0.5ex] 
    \midrule
        Value & 4 & 3.5 & 3 & 3 & 2 & 6 & 0.21 \\
        [0.5ex]
    \bottomrule
    \end{tabular}
    \begin{tablenotes}
        \footnotesize
        \item{This table shows some statistical measures to quantify the wealth of class \(C\).}
    \end{tablenotes}
    \end{threeparttable}
\end{center}

In addition, Lorenz curve for wealth of entities in class \(C\) is shown in \autoref{fig:sample-lorenz}.

\begin{figure}[!h]
    \centering
    \includegraphics[scale=0.8]{Sample Lorenz Curve}
    \caption{Lorenz curve of class \(C\)} \label{fig:sample-lorenz}
\end{figure}
\subsection{Python-Based Model Implementation}

Our study provides a Python-based implementation library for both the formal model and the insight model. The library incorporates the three notions of knowledge wealth discussed in \autoref{wealth-formal-model}, as well as the insight model outlined in \autoref{insight-model}. While the model is platform-agnostic, i.e., it can be applied to any KG, we focus its implementation on Wikidata for demonstrating its use cases. Wikidata is specifically chosen because of the availability of Wikidata Query Service which  facilitates structured queries over its data.

The notions of wealth are implemented at the query level, meaning that the filters and aggregations based on each wealth definition are directly embedded in SPARQL queries. The flow begins with defining class filters to specify the entities of interest. Next, SPARQL querying is performed using Wikidata Query Service to retrieve RDF triples matching the specified criteria and aggregate them according to the selected notion of wealth. The extracted data is then structured into pandas DataFrames, enabling further analysis through the insight model using Python libraries.

In our implementation, we only consider direct properties and exclude blank nodes to reduce complexity and maintain data quality. The complete flow of the formal model and insight model usage is illustrated in \autoref{fig:model-implementation}.

\begin{figure}[!htbp]
    \centering
    % \includegraphics[scale=.5]{Gini - Pure Literal}
    \makebox[\textwidth][c]{\includegraphics[width=1\textwidth]{Model Implementation.pdf}}%
    \caption{Model implementation and evaluation flow using Python Jupyter Notebook} \label{fig:model-implementation}
\end{figure}

%% application of formal model on WD
\section{Use Cases and Evaluation}

% \subsection{Bias in Wikidata}
\subsection{Group-Level Gap in Representation in Wikidata}

In this subchapter, analysis is conducted to see whether any particular entity group in Wikidata is underrepresented compared to others. There are 2 analysis done: gender-based gap and regional gap.


% \paragraph{Gender Bias in Wikidata.}
\paragraph{Gender-Based Knowledge Gap: Male vs. Female Representation.}
Gender-based knowledge gap analysis in Wikidata will be performed on 10 Wikidata classes: computer scientist, American singer, American actress/actor, badminton player, businessperson, lawyer, American politician, American writer, American researcher, and American journalist. These classes are chosen to ensure variation in occupations, as gender representation differs across professions. Additionally, six of these classes are limited to American entities to manage scalability, as querying large, global datasets can lead to excessive query time and computational resource constraints. The United States, being a large and well-known country, is expected to provide a broadly reflective sample while keeping the analysis computationally feasible.

To analyze the gap, the first aspect that will be considered is the proportion of each gender in every class. We assumed that there are equal numbers of males and females in real-world and this will be the basis to determine if there is any gap in the data. Pearson's chi-square test (goodness-of-fit) is then performed to test the null and alternative hypotheses with significance level of \(\alpha=\)5\% as follows:

% \(H_0\): The proportions of males and females in a particular class are equal to the real-world proportion

% \(H_1\): The proportions of males and females in a particular class are not equal to the real-world proportion

\begin{table}[h]
    \centering
    \renewcommand{\arraystretch}{1.3}
    \begin{tabular}{|l p{12cm}|} 
        \hline
        \multicolumn{2}{|l|}{\textbf{Gender-Based Knowledge Gap: Entity Count Gap (Pearson's chi-square test)}} \\
        \hline
        \textbf{$H_0$} & The proportions of males and females in a particular class are equal to the assumed real-world proportion of 50\%-50\%. \\
        \textbf{$H_1$} & The proportions of males and females in a particular class are not equal to the assumed real-world proportion of 50\%-50\%. \\
        \hline
        \textbf{Insight 1:} & Across all 10 classes, male entities significantly outnumber female entities in all classes. This suggests a systematic underrepresentation of female entities in Wikidata. \\
        \hline
    \end{tabular}
\end{table}

% \vspace{-3em}

\begin{center}
    \scriptsize
    \makebox[\linewidth]{
    \begin{threeparttable}
    \captionsetup{font=small}
    \caption{Entity Count of 10 Wikidata Classes per Gender Category}
    \label{tab:gender - entity count}
    \begin{tabular}{c | c c c c c | c c} 

    \toprule
        Class Name & Entity & Male & Female & \%Male & \%Female & $\chi^2$ & p-value \\ [0.5ex] 

    \midrule
        American actress/actor & 38655 & 21787 & 16868 & 0.563627 & 0.436373 & 625.96 & 3.78e-138 \\
        American journalist & 18033 & 12402 & 5631 & 0.687739 & 0.312261 & 2542.36 & 0.0 \\
        American politician & 96507 & 85800 & 10707 & 0.889055 & 0.110945 & 58430.57 & 0.0 \\
        American researcher & 5233 & 3634 & 1599 & 0.694439 & 0.305561 & 791.37 & 4.06e-174 \\
        American singer & 16038 & 9198 & 6840 & 0.573513 & 0.426487 & 346.69 & 2.23e-77 \\
        American writer & 33554 & 19656 & 13898 & 0.585802 & 0.414198 & 988.10 & 6.95e-217 \\
        Computer scientist & 19049 & 16180 & 2869 & 0.849388 & 0.150612 & 9301.42 & 0.0 \\
        Badminton player & 25427 & 13493 & 11934 & 0.530656 & 0.469344 & 95.59 & 1.42e-22 \\
        Businessperson & 76758 & 68583 & 8175 & 0.893496 & 0.106504 & 47540.67 & 0.0 \\
        Lawyer & 94479 & 83147 & 11332 & 0.880058 & 0.119942 & 54587.73 & 0.0 \\ [1ex]
    \bottomrule

    \end{tabular}
    \begin{tablenotes}
        \scriptsize
        \item{This table shows the entity count of 10 Wikidata classes per Gender Category. Chi-square test result shows the significance of difference between the entity count of the two genders male and female.}
    \end{tablenotes}
    \end{threeparttable}
    }
\end{center}

From \autoref{tab:gender - entity count}, we can see that there are more male entities than female entities in all of the classes. In terms of entity count, the gender gaps in some classes such as American singer, American actress/actor, badminton player, and American writer, are slim. The gender gaps in some other classes are huge, and it can be observed in the classes of computer scientist, businessperson, lawyer, American politician, journalist, and researcher. This phenomenon can also be easily identified through visualization, as exhibited in \autoref{fig:bias histogram-computer scientist}, where the histogram of the female subclass is much smaller compared to the male. Looking at the chi-square test result, as p-value is well below the chosen significance level, the null hypothesis is rejected in all classes. Hence, we considered the difference of entity count to be significant and conclude that the proportions of males and females in each Wikidata class are not the same as the assumed real-world proportion of 50\%-50\%.

However, it is arguable that, for some classes, the gap in entity count between both genders is expected because, in reality, there are more men than women in the workforce, especially in particular fields such as engineering. As a consequence, it is not reasonable if we expect to have an equal number of males and females entities in Wikidata. Therefore, entity count may not be a good measure of gap because of the nature of the data itself. To address this, we need to evaluate other metrics which can quantify the gap at entity-level.

The next metrics to be considered are the measures of central tendency and dispersion to see where the wealth distribution is concentrated and how the data spread.

\begin{table}[h]
    \centering
    \renewcommand{\arraystretch}{1.3}
    \begin{tabular}{|l p{12cm}|} 
        \hline
        \multicolumn{2}{|l|}{\textbf{Gender-Based Knowledge Gap: Measures of Central Tendency and Dispersion}} \\
        \hline
        \textbf{Insight 1:} & For all 10 classes, female values for mean, median, and mode are generally lower than those of males. Only in the American Singer class does a female (Madonna) appear as the richest individual. \\
        \textbf{Insight 2:} & Positive value of skewness and high value of kurtosis are observed across the board, indicating right-skewed distribution and frequent extreme outliers. \\
        \hline
    \end{tabular}
\end{table}

% \vspace{-3em}

\begin{center}
    \scriptsize
    \makebox[\linewidth]{
    \begin{threeparttable}
    \captionsetup{font=small}
    \caption{Measures of Central Tendency of 10 Wikidata Classes per Gender Category}
    \label{tab:gender - central tendency}
    \begin{tabular}{c | c c c} 

    \toprule
        Class Name & \CellWithForceBreak{Mean \\ (o/m/f)} & \CellWithForceBreak{Median \\ (o/m/f)} & \CellWithForceBreak{Mode \\ (o/m/f)} \\ [0.5ex] 
    \midrule
        American actress/actor & 39.72/40.82/38.31 & 30.00/30.00/28.00 & 19/19/19 \\
        American journalist & 31.30/33.12/27.29 & 24.00/25.00/21.00 & 14/15/14 \\
        American politician & 19.15/19.31/17.90 & 15.00/15.00/15.00 & 9/9/12 \\
        American researcher & 24.24/25.32/21.77 & 20.00/21.00/19.00 & 12/12/15 \\
        American singer & 42.96/43.30/42.51 & 31.00/33.00/30.00 & 18/24/15 \\
        American writer & 39.75/44.05/33.66 & 30.00/33.00/26.00 & 19/21/19 \\
        Computer scientist & 24.30/24.65/22.33 & 19.00/19.00/18.00 & 8/8/11 \\
        Badminton player & 21.90/21.64/22.19 & 16.00/16.00/16.00 & 14/14/14 \\
        Businessperson & 17.06/16.98/17.73 & 13.00/13.00/13.00 & 10/10/9 \\
        Lawyer & 22.58/23.16/18.33 & 19.00/19.00/15.00 & 14/16/12 \\
        [1ex]
    \bottomrule
    \end{tabular}
    \begin{tablenotes}
        \scriptsize
        \item{This table shows the measures of central tendency of 10 Wikidata classes per gender category. Each measure will have 3 values: o (overall), m (male), and f (female).}
    \end{tablenotes}
    \end{threeparttable}
    }
\end{center}

% \vspace{-2em}

\begin{center}
    \scriptsize
    \makebox[\linewidth]{
    \begin{threeparttable}
    \captionsetup{font=small}
    \caption{Measures of Dispersion and Symmetry of 10 Wikidata Classes per Gender Category}
    \label{tab:gender - dispersion and symmetry}
    \begin{tabular}{c | c c c c c} 

    \toprule
        Class Name & \CellWithForceBreak{Min \\ (o/m/f)} & \CellWithForceBreak{Max \\ (o/m/f)} & \CellWithForceBreak{Std. Deviation \\ (o/m/f)} & \CellWithForceBreak{Skewness \\ (o/m/f)} & \CellWithForceBreak{Kurtosis \\ (o/m/f)} \\ [0.5ex] 
    \midrule
        American actress/actor & 4/4/4 & 703/585/703 & 35.15/34.47/35.97 & 4.22/3.65/4.88 & 30.55/22.97/39.11 \\
        American journalist & 4/4/4 & 418/418/363 & 27.78/29.22/23.82 & 4.21/4.17/4.15 & 30.60/29.94/29.09 \\
        American politician & 4/4/4 & 564/564/334 & 15.05/15.07/14.90 & 6.98/7.01/6.83 & 112.45/115.94/84.91 \\
        American researcher & 4/4/4 & 225/225/195 & 17.16/18.46/13.46 & 3.86/3.75/3.74 & 25.74/23.07/31.65 \\
        American singer & 4/5/4 & 703/585/703 & 40.36/36.61/44.90 & 4.15/3.33/4.65 & 29.66/19.78/33.57 \\
        American writer & 4/4/4 & 564/564/435 & 35.82/39.48/28.82 & 3.63/3.37/4.02 & 21.27/18.08/27.34 \\
        Computer scientist & 3/3/3 & 452/452/178 & 19.69/20.35/15.35 & 3.48/3.52/2.25 & 28.77/28.90/9.89 \\
        Badminton player & 9/9/9 & 360/240/360 & 15.98/15.15/16.88 & 4.28/3.94/4.52 & 31.54/23.57/36.62 \\
        Businessperson & 3/3/3 & 585/585/434 & 15.06/14.39/19.81 & 8.07/7.80/8.17 & 142.31/144.95/106.75 \\
        Lawyer & 3/3/3 & 608/608/334 & 17.02/17.34/13.73 & 5.78/5.79/5.67 & 82.87/82.78/78.35 \\[1ex]
    \bottomrule
    \end{tabular}
    \begin{tablenotes}
        \scriptsize
        \item{This table shows the measures of dispersion and symmetry of 10 Wikidata classes per gender category. Each measure will have 3 values: o (overall), m (male), and f (female).}
    \end{tablenotes}
    \end{threeparttable}
    }
\end{center}

From \autoref{tab:gender - central tendency}, female entities generally have lower values of measure of central tendency (mean, median, mode). These characteristics can also be observed from the histogram in \autoref{fig:bias histogram-computer scientist}: female histograms’ peak and dense area are located on the left of the male’s. However, there are some classes in which the richest entity is a female. An example for this is the class of American Singer, which is shown by \autoref{fig:bias histogram-american singer}. Though the value of mean, median, and mode of count of properties are lower for female compared to male, the richest entity on that class is a female entity \textit{Madonna} (Q1744) with bag of property count of 703, with a significant difference with \textit{Michael Jackson} (Q2831) with bag of property count of 585.

From \autoref{tab:gender - dispersion and symmetry}, we also observed positive values of skewness (skewness > 0) and high kurtosis values (kurtosis > 3) in all classes, denoting the wealth distribution is right-skewed and leptokurtic. The high skewness values indicate that a small number of individuals accumulate disproportionately large property counts. Moreover, high kurtosis values across all genders suggest wealth distributions with heavy tails, meaning that extreme outliers are relatively frequent. Furthermore, we saw higher variability among males, as reflected in their generally wider range of property counts and higher standard deviations. These statistical properties point toward a more unequal wealth distribution among male entities, while females tend to cluster within a lower and more compressed wealth range.

\begin{figure}[!h]
\centering 
\subfloat[Histogram and Marginal Distribution Plot of Wealth for Class Computer Scientist\label{fig:bias histogram-computer scientist}]{%
  \includegraphics[clip,width=1.0\columnwidth]{Histogram of Wealth - Computer Scientist}%
}

\subfloat[Histogram and Marginal Distribution Plot of Wealth for Class American Singer\label{fig:bias histogram-american singer}]{%
  \includegraphics[clip,width=1.0\columnwidth]{Histogram of Wealth - American Singer}%
}

\caption{Histogram of wealth} \label{fig:Histogram of Wealth}

\end{figure}

At a glance we saw female classes are poorer compared to the male classes. To test this, we used t-test and Welch's test. First, we performed F-test to check if the male and female classes have equal variance. The result of F-test is then used to determine the appropiate test to be used in each class. Those with equal variance, i.e., if the p-value is more than 0.05, will use t-test; otherwise Welch's test is used. Then, we performed the appropiate tests to verify the null and alternative hypotheses with significance level of \(\alpha=5\%\) as follows:

% \(H_0\): The means of wealth of males and females in a particular class are equal

% \(H_1\): The means of wealth of males and females in a particular class are not equal

\begin{table}[h!]
    \centering
    \renewcommand{\arraystretch}{1.3}
    \begin{tabular}{|l p{12cm}|} 
        \hline
        \multicolumn{2}{|l|}{\textbf{Gender-Based Knowledge Gap: Mean of Wealth Gap (two-sided t-test and Welch's test)}} \\
        \hline
        \textbf{$H_0$} & The means of wealth of males and females in a particular class are equal. \\
        \textbf{$H_1$} & The means of wealth of males and females in a particular class are not equal. \\
        \hline
        \textbf{Insight 1:} & 9 out of 10 classes shows unequal mean, with 7 of them are in favor of males. \\
        \hline
    \end{tabular}
\end{table}

% \vspace{-3em}

\begin{center}
    \scriptsize
    \makebox[\linewidth]{
    \begin{threeparttable}
    \captionsetup{font=small}
    \caption{F-Test, t-Test, and Welch's Test Result of 10 Wikidata Classes}
    \label{tab:gender - mean test}
    \begin{tabular}{c | c c | c c | c c} 

    \toprule
        Class Name & \CellWithForceBreak{F-Test \\ statistic} & \CellWithForceBreak{F-Test \\ p-value} & \CellWithForceBreak{t-Test \\ statistic} & \CellWithForceBreak{t-Test \\ p-value} & \CellWithForceBreak{Welch's Test \\ statistic} & \CellWithForceBreak{Welch's \\ p-value} \\ [0.5ex] 

    \midrule
        American actress/actor & 0.92 & 0.00 & - & - & 6.93 & 4.27e-12 \\
        American journalist & 1.50 & 1.00 & 13.12 & 4.05e-39 & - & - \\
        American politician & 1.02 & 0.94 & 9.11 & 8.46e-20& - & - \\
        American researcher & 1.88 & 1.00 & 6.92 & 4.91e-12 & - & - \\
        American singer & 0.66 & 0.00 & - & - & 1.19 & 0.23 \\
        American writer & 1.88 & 1.00 & 26.44 & 1.98e-152 & - & - \\
        Computer scientist & 1.76 & 1.00 & 5.83 & 5.80e-09 & - & - \\
        Badminton player & 0.81 & 0.00 & - & - & -2.75 & 0.01 \\
        Businessperson & 0.53 & 0.00 & - & - & -3.34 & 0.00 \\
        Lawyer & 1.60 & 1.00 & 28.47 & 1.69e-177 & - & - \\
    \bottomrule

    \end{tabular}
    \begin{tablenotes}
        \scriptsize
        \item{This table shows the result of F-test, t-test, and Welch's test of 10 Wikidata classes to compare the significance difference between the males and females within each class.}
    \end{tablenotes}
    \end{threeparttable}
    }
\end{center}

Based on F-test on the 10 classes, variance between male and female subclasses was found to be unequal (p < 0.05) in 4 classes, requiring the use of Welch’s test. From the test results in \autoref{tab:gender - mean test}, we rejected the null hypothesis in 9 out of 10 class--American singer being the only exception with p-value > 0.05. Among them, 7 classes' means are in favor of male. The other 2 classes, badminton player and businessperson, deviate from this trend, where Welch’s test results indicate that the female subclasses actually have a higher mean wealth. The extremely small p-values in classes such as American writer and lawyer indicate a very strong statistical signal that the mean wealth differs significantly between genders. Finally, we may conclude that female classes are more likely to have smaller means than male classes, although certain classes may reflect localized or domain-specific representation imbalances. These results reinforce our earlier observations that female entities tend to be poorer in terms of knowledge wealth.

Entity count alone might not be a reliable measure of gap because the nature of the data itself. For instance, in real world, the number of male computer scientists is greater than that of female computer scientists. Therefore, expecting an equal number of male and female entities in Wikidata for this class would be unreasonable. Similarly, measures of central tendency—such as the mean or median—may offer a general overview but fail to capture the distributional nuances of individual entities within a class. To address this, a more holistic metric is needed.

Here, we introduced a new measure: the ratio of top \(x\)\% representation (of male/female) relative to expectation. The value of expectation of a gender in a class is equal to the percentage of that particular in the class. Top \(x\)\% male relative to expectation is the ratio of percentage of male entities in the top \(x\)\% to the expectation. Similarly, top \(x\)\% female relative to expectation is the ratio of percentage of female entities in the top \(x\)\% to the expectation.

\begin{table}[h]
    \centering
    \renewcommand{\arraystretch}{1.3}
    \begin{tabular}{|l p{12cm}|} 
        \hline
        \multicolumn{2}{|l|}{\textbf{Gender-Based Knowledge Gap: Relative Expectation Ratios}} \\
        \hline
        \textbf{Insight 1:} & A ratio value of 1 indicates a balanced distribution, where the group's actual share of wealth aligns with its expected share under equal representation. \\
        \textbf{Insight 2:} & On average across all 8 classes, the bottom six quantiles (i.e., the lower wealth segments) are predominantly composed of female entities, while the top four quantiles (i.e., the wealthier segments) are predominantly male. \\
        \textbf{Insight 3:} & At the individual class level, 6 out of 8 classes show male dominance in the upper wealth quantiles. The exceptions are the businessperson and badminton player classes. \\
        \hline
    \end{tabular}
\end{table}

% \vspace{-2em}

\begin{figure}[!h]
    \centering 
    \subfloat[Ratio of Class Wealth to Expectation per Cumulative Top Percentage - All Classes Average
    \label{fig:test1}]{%
      \includegraphics[clip,width=1.0\columnwidth]{Ratio of Class Wealth to Expectation per Cumulative Top Percentage - All Classes Average - Gender}%
    }
    
    \subfloat[Ratio of Class Wealth to Expectation per Quantile - All Classes Average\label{fig:test2}]{%
      \includegraphics[clip,width=1.0\columnwidth]{Ratio of Class Wealth to Expectation per Quantile - All Classes Average - Gender}%
    }
    
    \caption{Ratio of each gender wealth to expectaion} \label{fig:gender - ratio of gender wealth to expectation}
    
\end{figure}

When the shape of distribution of male and female of a class is the same (in other word, the wealth is distributed equivalently to male and female entities), then the value of top \(x\)\% relative to expectation should be 1 for both male and female subclasses. A value higher than 1 indicates domination by that particular gender. Conversely, a value lower than 1 indicates underrepresentation.

From \autoref{fig:gender - ratio of gender wealth to expectation} the value of ratio between top \(x\)\% potion to the expectation in the above tables, we can see that on average, the rich entities are dominated by male. Exceptions are held for 2 classes, that is classes businessperson, and badminton player. In the businessperson class, female dominance in the top quantile (Q10) creates an exception. However, quantiles 2–9 are predominantly male-dominated, albeit only slightly and very close to the balance line. Meanwhile, the lowest quantile (Q1) is significantly female-dominated. The badminton player class deviates from all other classes by exhibiting a distinct zig-zag pattern across quantiles, indicating alternating dominance between males and females, without a clear upward or downward trend.

Moreover, as we set bigger portions (higher percentage), the gap of ratio between the two ender in each class decreases i.e. the value of top \(x\)\% relative to expectation of both genders converge to 1.
\paragraph{Western Bias in Wikidata}

Western bias analysis in Wikidata will be performed on 5 Wikidata classes: computer scientist, singer, memorial, university, and river. For each class, we collect the data for the western portion from 8 countries: Canada, France, Germany, Ireland, Poland, Switzerland, the United Kingdom (UK), and the United State of America (USA). For the non-western portion, we also chose 8 countries: China, Egypt, India, Indonesia, Japan, Morocco, Nigeria, and South Africa.

To analyze the bias, the first aspect that will be considered is the proportion of each regional category in every class. We assume that there are equal numbers of western and non-western and this will be the basis to determine if there is any bias in the data. Pearson's chi-square test (goodness-of-fit) is then performed to test the null and alternative hypotheses with significance level of \(\alpha=5\%\) as follows:

\(H_0\): The proportions of western and non-western entities in a particular class are equal

\(H_1\): The proportions of western and non-western entities in a particular class are not equal


\begin{center}
\small
\begin{threeparttable}
\caption{Entity Count of 5 Wikidata Classes per Regional Category}
\label{tab:western - entity count}
\begin{tabular}{c | c c c c c c c} 

\toprule
    Class Name & Entity & Western & \CellWithForceBreak{Non- \\ western} & \%Western & \CellWithForceBreak{\%Non- \\ western}& $\chi^2$ & p-value \\ [0.5ex] 
\midrule
    Computer scientist & 6063 & 5446 & 617 & 0.90 & 0.10 & 3846.15 & 0.0 \\
    Singer & 43240 & 31039 & 12201 & 0.72 & 0.28 & 8206.99 & 0.0 \\
    Memorial & 4011 & 3836 & 175 & 0.96 & 0.04 & 3341.54 & 0.0 \\
    University & 6124 & 2398 & 3726 & 0.39 & 0.61 & 287.98 & 1.37e-64 \\
    River & 125567 & 70059 & 55508 & 0.56 & 0.44 & 1686.20 & 0.0 \\
    [1ex]
\bottomrule
\end{tabular}
\begin{tablenotes}
    \footnotesize
    \item{This table shows the entity count of 5 Wikidata classes per regional category. Chi-square test result shows the significance of difference between the entity count of the two genders male and female.}
\end{tablenotes}
\end{threeparttable}
\end{center}

In terms of entity count, \autoref{tab:western - entity count} shows that there are big gaps between the westerners and non-westerners in all of the classes. From \autoref{tab:western - central tendency}, non-western entities generally have lower values of measure of central tendency (mean, median, mode). The range of property count of non-westerns is generally also lower than the westerns. Positive values of skewness (skewness > 0) and high kurtosis values (kurtosis > 3) in all classes denote the wealth distribution is right skewed and leptokurtic.

\begin{center}
\small
\begin{threeparttable}
\caption{Measures of Central Tendency of 5 Wikidata Classes per Regional Category}
\label{tab:western - central tendency}
\begin{tabular}{c | c c c} 
\toprule
    Class Name & \CellWithForceBreak{Mean \\ (o/w/n)} & \CellWithForceBreak{Median \\ (o/w/n)} & \CellWithForceBreak{Mode \\ (o/w/n)} \\ [0.5ex] 
\midrule
    Computer scientist & 35.00/35.87/27.24 & 29.00/29.00/22.00 & 21/15/16 \\
    Singer & 34.99/39.14/24.43 & 25.00/29.00/18.00 & 15/18/14 \\
    Memorial & 11.04/11.04/11.13 & 9.00/9.00/9.00 & 9/9/9 \\
    University & 23.11/31.61/17.63 & 17.50/24.00/16.00 & 6/6/6 \\
    River & 7.69/8.55/6.60 & 7.00/7.00/6.00 & 7/7/7 \\
    [1ex]
\bottomrule
\end{tabular}
\begin{tablenotes}
    \footnotesize
    \item{This table shows the measures of central tendency of 5 Wikidata classes per regional category. Each measure will have 3 values: o (overall), w (western), and f (non-western).}
\end{tablenotes}
\end{threeparttable}
\end{center}

\begin{center}
\small
\begin{threeparttable}
\caption{Measures of Dispersion and Symmetry of 5 Wikidata Classes per Gender Category}
\label{tab:western - dispersion and symmetry}
\begin{tabular}{c | c c c c c} 

\toprule
    Class Name & \CellWithForceBreak{Min \\ (o/w/n)} & \CellWithForceBreak{Max \\ (o/w/n)} & \CellWithForceBreak{Std. Deviation \\ (o/w/n)} & \CellWithForceBreak{Skewness \\ (o/w/n)} & \CellWithForceBreak{Kurtosis \\ (o/w/n)} \\ [0.5ex] 
\midrule
    Computer scientist & 4/4/5 & 441/441/145 & 25.15/25.67/18.15 & 3.02/3.02/2.37 & 20.90/20.83/8.49 \\
    Singer & 3/4/3 & 687/687/379 & 33.27/36.60/18.94 & 4.49/4.28/3.15 & 35.56/31.09/21.59 \\
    Memorial & 2/3/2 & 142/142/52 & 5.89/5.82/7.28 & 8.50/8.95/2.98 & 143.57/156.15/12.79 \\
    University & 2/2/2 & 234/234/166 & 20.00/25.88/12.25 & 2.37/1.65/2.22 & 10.34/5.33/12.70 \\
    River & 2/2/2 & 452/452/148 & 5.24/6.46/2.68 & 21.71/19.54/14.67 & 1152.84/868.56/465.42 \\
    [1ex]
\bottomrule
\end{tabular}
\begin{tablenotes}
    \footnotesize
    \item{This table shows the measures of dispersion and symmetry of 5 Wikidata classes per gender category. Each measure will have 3 values: o (overall), w (western), and f (non-western).}
\end{tablenotes}
\end{threeparttable}
\end{center}

Out of 5 classes, class of memorial is the only class where the null hypothesis is not rejected. In the other 4 classes, we can see a significant difference between the mean of the two reginal categories, which all are in favor of the western.

\begin{center}
\small
\begin{threeparttable}
\caption{F-Test, T-Test, and Welch's Test Result of 5 Wikidata Classes}
\label{tab:western - mean test}
\begin{tabular}{c | c c c c c c} 
\toprule
    Class Name & \CellWithForceBreak{F-Test \\ statistic} & \CellWithForceBreak{F-Test \\ p-value} & \CellWithForceBreak{T-Test \\ statistic} & \CellWithForceBreak{T-Test \\ p-value} & \CellWithForceBreak{Welch's Test \\ statistic} & \CellWithForceBreak{Welch's \\ p-value} \\ [0.5ex] 
\midrule
    Computer scientist & 2.00 & 1.00 & 8.13 & 5.10e-16 & 10.66 & 3.64e-25 \\
    Singer & 3.73 & 1.00 & 42.22 & 0.0 & 54.59 & 0.0 \\
    Memorial & 0.64 & 0.00 & -0.19 & 0.85 & -0.16 & 0.88 \\
    University & 4.46 & 1.00 & 28.40 & 8.23e-167 & 24.73 & 2.12e-123 \\
    River & 5.83 & 1.00 & 66.91 & 0.0 & 72.64 & 0.0 \\
 [1ex]
\bottomrule
\end{tabular}
\begin{tablenotes}
    \footnotesize
    \item{}
\end{tablenotes}
\end{threeparttable}
\end{center}

\begin{figure}
\centering 
\subfloat[Ratio of Class Wealth to Expectation per Cumulative Top Percentage - All Classes Average
\label{fig:test1 - western}]{%
  \includegraphics[clip,width=1.0\columnwidth]{Ratio of Class Wealth to Expectation per Cumulative Top Percentage - All Classes Average - Region}%
}

\subfloat[Ratio of Class Wealth to Expectation per Quantile - All Classes Average\label{fig:test2 - western}]{%
  \includegraphics[clip,width=1.0\columnwidth]{Ratio of Class Wealth to Expectation per Quantile - All Classes Average - Gender - Region}%
}

\caption{Ratio of each regional wealth to expectaion}\label{fig:western - ratio of regional wealth to expectation}

\end{figure}
\subsection{Effect of Type of Wealth on Inequality Measure} \label{wealth & gini}

% \begin{center}
    \small
    \begin{threeparttable}
    \caption{Gini Bag-Set}
    \label{tab:gini bag-set}
    \begin{tabular}{c c c} 
    
    \toprule
        Class Name & Gini Bag & Gini Set \\ [0.5ex] 
    \midrule
        Historical painting & 0.22 & 0.14 \\
        University & 0.44 & 0.41 \\
        Sci-fi book & 0.32 & 0.22 \\
        Memorial & 0.22 & 0.18 \\
        American researcher & 0.33 & 0.31 \\
        American singer & 0.40 & 0.39 \\
        Badminton player & 0.29 & 0.14 \\
        Computer scientist & 0.41 & 0.37 \\
        [1ex]
    \bottomrule
    \end{tabular}
    \begin{tablenotes}
        \footnotesize
        This table shows the measures of gini coefficient of knowledge wealth based on the (non-)uniqueness of individual properties of 8 Wikidata classes
    \end{tablenotes}
    \end{threeparttable}
\end{center}
% \begin{center}
    \small
    \begin{threeparttable}
    \caption{Gini Object-Literal-ID}
    \label{tab:gini proptype}
    \begin{tabular}{c c c c} 
    
    \toprule
        Class Name & Gini Object & Gini Literal & Gini ID \\ [0.5ex] 
    \midrule
        American researcher & 0.26 & 0.50 & 0.50 \\
        American singer & 0.29 & 0.50 & 0.53 \\
        Badminton player & 0.30 & 0.36 & 0.60 \\
        Computer scientist & 0.36 & 0.54 & 0.56 \\
        Historical painting & 0.25 & 0.27 & 0.44 \\
        Memorial & 0.20 & 0.30 & 0.40 \\
        Sci-fi book & 0.35 & 0.34 & 0.42 \\
        University & 0.40 & 0.49 & 0.53 \\
        [1ex]
    \bottomrule
    \end{tabular}
    \begin{tablenotes}
        \footnotesize
        This table shows the measures of gini coefficient of knowledge wealth based on types of property of 8 Wikidata classes
    \end{tablenotes}
    \end{threeparttable}
\end{center}
% \begin{center}
    \small
    \begin{threeparttable}
    \caption{Gini Outgoing-Incoming}
    \label{tab:gini outgoing-incoming}
    \begin{tabular}{c c c} 
    
    \toprule
        Class Name & Gini Outgoing & Gini Incoming \\ [0.5ex] 
    \midrule
        Historical painting & 0.22 & 0.86 \\
        University & 0.44 & 0.91 \\
        Sci-fi book & 0.32 & 0.82 \\
        Memorial & 0.22 & 0.99 \\
        American researcher & 0.33 & 0.77 \\
        American singer & 0.40 & 0.82 \\
        Badminton player & 0.29 & 0.68 \\
        Computer scientist & 0.41 & 0.81 \\
        [1ex]
    \bottomrule
    \end{tabular}
    \begin{tablenotes}
        \footnotesize
        This table shows the measures of gini coefficient of knowledge wealth based on the direction of link of 8 Wikidata classes
    \end{tablenotes}
    \end{threeparttable}
\end{center}

In this subchapter, analysis is done to see how each wealth type affects the level of inequality of Wikidata classes. There are 2 ways this is done--quantitatively using Gini coefficient and qualitatively using Lorenz curve. The analysis is performed on 8 Wikidata classes, in which 4 of them are human-related class while the other 4 are not.

\begin{center}
    \small
    \makebox[\linewidth]{
    \begin{threeparttable}
    \caption{Knowledge Wealth Type on Gini Coefficient}
    \label{table:gini-coef}
    \begin{tabular}{c | c c | c c c | c c} 
    
    \toprule
        Class Name & \CellWithForceBreak{Gini \\ Bag} & \CellWithForceBreak{Gini \\ Set} & \CellWithForceBreak{Gini \\ Object} & \CellWithForceBreak{Gini \\ Literal} & \CellWithForceBreak{Gini \\ ID} & \CellWithForceBreak{Gini \\ Outgoing} & \CellWithForceBreak{Gini \\ Incoming} \\ [0.5ex] 
    \midrule
        American researcher & 0.33 & 0.31 & 0.26 & 0.65 & 0.50 & 0.33 & 0.77 \\
        American singer & 0.40 & 0.39 & 0.29 & 0.36 & 0.53 & 0.40 & 0.82  \\
        Badminton player & 0.29 & 0.14 & 0.30 & 0.14 & 0.60 & 0.29 & 0.68 \\
        Computer scientist & 0.41 & 0.37 & 0.36 & 0.64 & 0.56 & 0.41 & 0.81 \\
        Historical painting & 0.23 & 0.15 & 0.25 & 0.26 & 0.45 & 0.23 & 0.87 \\
        Memorial & 0.22 & 0.19 & 0.20 & 0.31 & 0.41 & 0.22 & 0.99 \\
        Sci-fi book & 0.30 & 0.21 & 0.31 & 0.33 & 0.44 & 0.30 & 0.82 \\
        University & 0.44 & 0.41 & 0.39 & 0.49 & 0.53 & 0.44 & 0.91 \\
        [1ex]
    \bottomrule
    \end{tabular}
    \begin{tablenotes}
        \footnotesize
        \item{This table shows the comparison of Gini coefficient of 8 Wikidata classes}
    \end{tablenotes}
    \end{threeparttable}
    }
\end{center}

When looking at the notion of wealth using the characteristics of (non-)uniqueness of individual properties, it is intuitive that the measure of bag of properties will always give higher (or at least, equal) amount of wealth compared to the measure of set. Set of property will have an upper bound of number of unique property, while bag of properties does not have any upper bound. Moreover, using bag of properties, a large number of triples having the same property may inflate the wealth substantially--though this is not necessarily a problem nor an advantage. This characteristics has a direct impact on inequality measure and it is well depicted on the value of Gini coefficient. From \autoref{table:gini-coef}, in all clasess, the Gini coefficient using bag of properties is always higher than of set of properties.

% - object, literal, ID: object tends to have lower inequality
Using the notion of wealth by type of property, in general the smallest Gini coefficient value is mostly from wealth using object property. The second one is when using literal, while the biggest one always comes when using ID. When we look at the contribution of each type of property to the overall wealth, 

\begin{center}
    \small
    \makebox[\linewidth]{
    \begin{threeparttable}
    \caption{Contribution of Property Type to Knowledge Wealth Using Bag of Properties}
    \label{table:prop-contribution-bag}
    \begin{tabular}{c | c c c | c | c} 
    
    \toprule
        Class Name & \CellWithForceBreak{\% Object \\ Bag} & \CellWithForceBreak{\% Literal \\ Bag} & \CellWithForceBreak{\% ID \\ Bag} & \CellWithForceBreak{\% Literal + ID \\ Bag} & \CellWithForceBreak{\% Object + Literal \\ Bag} \\ [0.5ex] 
    \midrule
        American researcher & 59.55 / 65.65 & 3.32 / 2.8 & 37.12 / 31.55 & 40.45 / 34.35 & 62.88 / 68.45 \\
        American singer & 40.44 / 48.42 & 7.47 / 8.48 & 52.09 / 43.1 & 59.56 / 51.58 & 47.91 / 56.9 \\
        Badminton player & 79.34 / 78.96 & 10.14 / 11.68 & 10.52 / 9.36 & 20.66 / 21.04 & 89.48 / 90.64 \\
        Computer scientist & 59.36 / 65.93 & 4.19 / 3.71 & 36.44 / 30.36 & 40.64 / 34.07 & 63.56 / 69.64 \\
        Historical painting & 66.35 / 66.46 & 25.48 / 25.85 & 8.17 / 7.69 & 33.65 / 33.54 & 91.83 / 92.31 \\
        Memorial & 62.82 / 64.16 & 21.0 / 20.4 & 16.18 / 15.44 & 37.18 / 35.84 & 83.82 / 84.56 \\
        Sci-fi book & 62.96 / 62.62 & 14.44 / 15.78 & 22.59 / 21.6 & 37.04 / 37.38 & 77.41 / 78.4 \\
        University & 37.59 / 44.79 & 18.11 / 18.31 & 44.3 / 36.9 & 62.41 / 55.21 & 55.7 / 63.1 \\
        [1ex]
    \bottomrule
    \end{tabular}
    \begin{tablenotes}
        \footnotesize
        \item{This table shows the contribution of each property type of 8 Wikidata classes based on the notion of bag of properties. Each record has 2 values separated by "/". The first value is the contribution rate when calculated with total sum method, and the second one is when using the average of individual percentage method}
    \end{tablenotes}
    \end{threeparttable}
    }
\end{center}

\begin{center}
    \small
    \makebox[\linewidth]{
    \begin{threeparttable}
    \caption{Contribution of Property Type to Knowledge Wealth Using Set of Properties}
    \label{table:prop-contribution-set}
    \begin{tabular}{c | c c c | c | c} 
    
    \toprule
        Class Name & \CellWithForceBreak{\% Object \\ Set} & \CellWithForceBreak{\% Literal \\ Set} & \CellWithForceBreak{\% ID \\ Set} & \CellWithForceBreak{\% Literal + ID \\ Set} & \CellWithForceBreak{\% Object + Literal \\ Set} \\ [0.5ex] 
    \midrule
        American researcher & 51.25 / 59.69 & 3.97 / 3.29 & 44.78 / 37.02 & 48.75 / 40.31 & 55.22 / 62.98 \\
        American singer & 33.91 / 43.58 & 8.1 / 9.18 & 57.99 / 47.24 & 66.09 / 56.42 & 42.01 / 52.76 \\
        Badminton player & 71.75 / 74.13 & 13.79 / 13.9 & 14.46 / 11.97 & 28.25 / 25.87 & 85.54 / 88.03 \\
        Computer scientist & 50.53 / 60.54 & 4.98 / 4.27 & 44.49 / 35.19 & 49.47 / 39.46 & 55.51 / 64.81 \\
        Historical painting & 60.57 / 61.91 & 28.88 / 28.61 & 10.55 / 9.48 & 39.43 / 38.09 & 89.45 / 90.52 \\
        Memorial & 60.06 / 62.01 & 22.41 / 21.59 & 17.54 / 16.4 & 39.94 / 37.99 & 82.46 / 83.6 \\
        Sci-fi book & 59.36 / 61.91 & 13.86 / 14.59 & 26.78 / 23.5 & 40.64 / 38.09 & 73.22 / 76.5 \\
        University & 33.93 / 42.74 & 17.67 / 18.32 & 48.4 / 38.95 & 66.07 / 57.26 & 51.6 / 61.05 \\
        [1ex]
    \bottomrule
    \end{tabular}
    \begin{tablenotes}
        \footnotesize
        \item{This table shows the contribution of each property type of 8 Wikidata classes based on the notion of set of properties. Each record has 2 values separated by "/". The first value is the contribution rate when calculated with total sum method, and the second one is when using the average of individual percentage method}
    \end{tablenotes}
    \end{threeparttable}
    }
\end{center}

% - object: 
% - literal: analisis kenapa yang badminton beda sendiri (object > literal). perlu cek yang literal only tanpa ID (cek query vs hitung manual literala all - ID)
% - ID: ID itu kan menandakan apakah suatu entity terdaftar di suatu DB external lain. nah ini lebih sensitif terhadap popularitas entity. biasanya hanya entity yang populer banget yang  (aspek eksponensial, ketika dia populer dia terdaftar di banyak db dibandingkan yang medioker. ketika viral, dia kaya domina effect terdaftar lagi di yang lain2). semacam incoming link

% - outgoing, incoming: incoming shows very high Gini, this is because it is harder --> show the lorenz curve
Using the notion of wealth by the direction of the link, the Gini coefficient when using incoming link is always higher than using outgoing link. By inspecting the Lorenz curve, we can see that most entities do not have any incoming link, and only the small percentage of entites has some incoming link. \autoref{fig:gini-outgoin&incoming} shows the comparison of Lorenz curve of knowledge wealth based on the direction of the link from 3 Wikidata classes. The difference between the two is very significant, because in \autoref{fig:gini-outgoing} the Lorenz curves are closer to the perfect equality line, meanwhile in \autoref{fig:gini-incoming} the diagonal and the Lorenz curve almost form a right triangle which is very close to maximum inequality.

% add to discussion 
% - Wikidata = entity-centric knowledge graph, dimana bentuknya (s,p,o) dengan s sebagai subject of interest. jadi memang expected bahwa outgoing banyak 
% sedangkan incoming mostly justru 0.
% - sedrastis ini perbedaan incoming vs outgoing. incoming link is underestimated. given o, what is the (s, p)
% - kasih contoh: 2 entitas yang secara outgoing mirip, tapi incoming-nya jomplang (1 kaya 1 miskin)

\begin{figure}[!htbp]
    \centering 
    \subfloat[Lorenz curve of wealth using outgoing link
    \label{fig:gini-outgoing}]{%
      \includegraphics[clip,width=1.0\columnwidth]{Gini - Outgoing}%
    }
    
    \subfloat[Lorenz curve of wealth using incoming link\label{fig:gini-incoming}]{%
      \includegraphics[clip,width=1.0\columnwidth]{Gini - Incoming.png}%
    }
    
    \caption{Comparison of Lorenz curve of wealth based on the direction of link} \label{fig:gini-outgoin&incoming}
    
\end{figure}


\subsection{(Tentative)Proxy of Knowledge Wealth: Real-World Rankings}
- Popularity
- Movie IMDb
- GDP, economic wealth
- Personal net worth

\subsection{(Tentative) Basics of Knowledge Wealth Comparison}

\paragraph{Subclass-to-Subclass Wealth Comparison}
\paragraph{Subclass-to-Superclass Wealth Comparison}
\paragraph{Totally Different Classes Wealth Comparison}

\subsection{(Tentative) Knowledge Wealth Evolution using Wikidata History}

\subsection{(Tentative) Pareto Principle in Wikidata}

\subsection{(Tentative)}
This is additional, if possible. Based on WN notes to use statistical measures for machine learning

\section{Discussion}

% add to discussion 
% - Wikidata = entity-centric knowledge graph, dimana bentuknya (s,p,o) dengan s sebagai subject of interest. jadi memang expected bahwa outgoing banyak 
% sedangkan incoming mostly justru 0.
% - sedrastis ini perbedaan incoming vs outgoing. incoming link is underestimated. given o, what is the (s, p)
% - kasih contoh: 2 entitas yang secara outgoing mirip, tapi incoming-nya jomplang (1 kaya 1 miskin)
\paragraph{Generality of Framework}
The framework that is proposed in this study is applicable to any kind of knowledge graph. However, the library that we built for experimentation is specifically for Wikidata, because the query service, structure, and response is very unique for each knowledge graph. If further experimentation is to be conducted for other knowledge graph, then the library needs to be modified.

\paragraph{Issue of Incoming Link}
Wikidata is an entity-centric knowledge graph which means in the editathon efforts, the subject \(s\) is always be the subject of interest and the starting point to be edited by contributors. It is very unlikely that the opposite approach is done, that is, an object \((o)\) is given and a pair subject and property \((s, p)\) is to be searched. Not only from the contributors, the platform itself does not support the later view. This explains the phenomenon that we observed in \autoref{wealth & gini} regarding outgoing and incoming link. With existing subject-centric point of view, the number of new outgoing link introduced to the knowledge graph will grow in a much higher rate as opposed to incoming link. As a result, incoming link will be very rare, or even only present in certain entites.

\paragraph{Weight of Property}
%Content dump for weighted form

% It is intuitive that the measure of bag of properties will always give higher (or at least, equal) amount of wealth compared to the measure of set. Set of property will have an upper bound of number of unique property, while bag of property does not have any upper bound. Moreover, using bag of properties, a large number of triples having the same property may inflate the wealth substantially--though this is not necessarily a problem nor an advantage.

Set of properties might be more suitable if the main concern is the presence of properties, instead of the abundance of information it contains. We will take \textit{William Shakespeare} (Q692) in Wikidata as an example. It is reasonable if property like  \textit{date of birth} (P569) to be treated using set of properties, but for a well-known playwright and poet, we shall expect the property \textit{notable work} (P800) to incorporate all or most of his well-known works. If only few works registered in \(G\) despite he has dozens of works, then we might conclude the entity is poor. For this case, treating the property using the notion of set is not preferable because it will fail to capture the aforementioned poor condition, while blatantly using bag of properties might skew the wealth amount.

Due to this, the notion of (non-)uniqueness can be extended to a weighted form. The weight is given independently for each property and can be defined in such a way that is most appropriate for the nature of the property. Example of weight is inverse of median of property count of each entity.

Let \(h_{p_i}\) be the weight of property \(p_i\) in graph \(G\). Let \(N_{bag}(s,G,p_i)\) be a set that comprises all pair  \((p_i,o)\), that is, property \(p_i\) and an object \(o\) that is connected to \(s\). Then \(W_{weighted}(s, G)\) is the sum of \(N_{bag}(s,G,p_i)\) multiplied by the associated weight \(h_{p_i}\).

% \textit{how can we formalize wi to be the function of set(amount of pi in class C in G)??}

% \[
%     \forall p_i, h_i = f_i(....)
% \]
\[
    N_{bag}(s,G,p_i) = \{(p_i, o) | (s, p_i, o) \in G\}
\]
\[
    W_{weighted}(s, G) = \sum_i |N_{bag}(s,G,p_i)| * h_{p_i}
\]

Using \autoref{fig:wealth-weighted}, we can show how the notion of weighted wealth can be calculated. Let's define \(h_{p_1} = 1/median\) and \(h_{p_2} = 1\). For property \(p_1\), \(\{1, 2, 2, 4\}\) is the sorted amount of information contributed from \(p_1\) in each individual entity from \(s_1\) to \(s_4\), from which we get \(h_{p_1} = 1/median = 1/2\). With the above definition, \(W_{weighted}(s_1, G) = 2\), \(W_{weighted}(s_2, G) = 2\), \(W_{weighted}(s_3, G) = 3.5\), \(W_{weighted}(s_4, G) = 1.5\).

Another phenomena that should also be considered is that an entity might have several occurrences of the same property, but this property might just be 'trivial'. This is similar to a document having a lot of 'the' or 'a' (stopwords). Conversely, an entity might just have a single occurrence of a property, but it is a non-trivial one. Perhaps, the property is a highly relevant one for the entity's class. For example, \textit{time period} (P2348) for \textit{William Shakespeare} (Q692) will be a highly relevant one for prominent poets. However, the property \textit{sibling} (P3373) might not be too relevant for Shakespeare's career. One idea to handle such trivial is to regard those porperties using the notion of set of properties--thus we care only on its existence rather than its actual count--or, we can use threshold function to set a tolerance for how much of such trivial we allow.

% =====> OR (alternative of weighted, or generalized form of Wealth based on the (non-)uniqueness of individual properties)

% =====> \(T_i\) is a multiset \(T_i = ... \) -> isinya list semua kontribusi wealth dari property \(p_i\) pada seluruh entity \(S\) pada class \(C\) di graph \(G\) -> keuntungannya disini kita jadi bisa punya fungsi konstan, sehingga bisa catter definisi set, bag, dan weighted secara bersamaan.

Thus, the notion of (non-)uniqueness can be generalized to accommodate the bag, set, and weighted concept. Let \(s_1\), \(s_2\), ... \(s_m\) be \(m\) distinct entities in graph \(G\), which all of them collectively create a class \(C\). Let \(N_{bag}(s,G,p_i)\) be a set that comprises all pair of a particular property \(p_i\) and object, \((p_i,o)\), that is connected to \(s\). We define \(T_{C,p_i}\) a multiset consisted of the number of non-unique properties of each entity of \(C\) contributed from property \(p_i\), i.e., multiset of cardinality of \(N_{bag}(s,G,p_i)\) for all \(s\) in \(C\). For each \(p_i\), let \(f_{p_i}\) be a multivariate function with 2 arguments: \(T_{C,p_i}\) and \(N_{bag}(s,G,p_i)\). \(w_{p_i}\) is the amount of wealth of \(s\) attributed from property \(p_i\), which is calculated by function \(f_{p_i}\). Then \(W_{}(s, G)\) is the sum of \(w_{p_i}\).


\[
    N_{bag}(s,G,p_i) = \{(p_i,o) | (s_j, p_i, o) \in G\}
\]
\[
    T_{C,p_i} = \{|N_{bag}(s,G,p_i)|\ | s \in C\}
\]
\[
    w_{p_i} = f_{p_i}(T_{C,p_i}, |N_{bag}(s,G,p_i)|)
\]
\[
    W_{}(s, G) = \sum_i w_{p_i}
\]

Using \autoref{fig:wealth-weighted}, we can show how the generalized form can be utilized. From the definition, we get \(T_{C,p_1} = \{2, 2, 5, 1\}\) and \(T_{C,p_2} = \{1, 1, 1, 1\}\). Property \(p_1\) illustrates the former issue of trivial. Here, we define $f_{p_1}(x,y) = \begin{cases}
      x & \text{if }x \leq 1 \\
      \frac{x}{2} + 1 & \text{if }1 < x \leq 4 \\
      3 & \text{if }x > 4
    \end{cases}\, $.
For \(p_2\), we want to handle it using the notion of set of properties, thus we define it using a constant function \(f_{p_2}(x,y) = 1\).
With the above definition, \(W_{}(s_1, G) = 3\), \(W_{}(s_2, G) = 3\), \(W_{}(s_3, G) = 4\), \(W_{}(s_4, G) = 2\).

\paragraph{Entity Misclassification}
% - Bruno Darcet: 54 literal pure (1 date of birth + 53 elo rating)
% - Bruno Darcet (1st rank) & Daniel José Queraltó (2nd rank) ini dari profilnya sebenarnya bukan computer scientist. sepertinya ada issue missclassification pada occupation
% - Ada issue serupa dengan Computer Scientist. Yang terkaya Alexander Julien (Q61249179) bukan university researcher
% - Badminton Player, Yang richest (Dirk Stikker) bukan prominent badminton player tapi lebih ke politikus yang punya keterkaitan dengan badminton
% - suatu entitas bisa aja belong to multiple class (case Dirk Stikker). Bisa ada juga miss-classification (Bruno Darcet). Ada issue data quality, yang mengharuskan data cleaning

\section{Conclusions}

In this study, we introduce a novel framework for analyzing knowledge wealth in knowledge graphs, using Wikidata as a case study. By drawing an analogy between financial wealth and information richness, we develop 3 notions of knowledge wealth based on property cardinality, property type, and link direction. We demonstrate how these different definitions yield varying inequality levels, as reflected in measures such as the Gini coefficient and Lorenz curves. Our proposed metric—relative expectation ratio—offers a powerful and intuitive way to assess representation bias across different entity groups. Unlike raw entity counts or central tendency measures, this metric captures the distributional imbalances of information across quantiles, providing a more nuanced understanding of how knowledge is allocated within Wikidata.

- mention library
- link ke GitHub

Future work may explore several things in order to improve the knowledge wealth framework. At the moment, our evaluation is limited to Wikidata. Applying this framework to other KGs beyond Wikidata would be valuable to assess the framework generalizability and to further explore the general quality of KGs. Another promising direction is to explore \textit{property-weighted knowledge wealth}, where not all properties are treated equally. In the current framework, each property contributes uniformly to an entity’s wealth, regardless of its importance or informativeness. Incorporating weights—based on factors such as semantic significance, frequency of use, or relevance to the class—could offer a more nuanced and realistic estimation of an entity's information richness. Last but not least, it is also of our interest to explore the use of machine learning algorithms to classify entities into “rich” and “poor” groups based on their knowledge wealth. Such clustering could assist contributors in identifying which segments require additional enrichment efforts, enabling more targeted and efficient improvements to knowledge graph completeness.

- saat ini library kita open, framework-nya general. Jadi harapannya bisa di-extend ke KGs lain. dengan cara ganti endpoint dan query
- di bagian machine learing, bisa bahas "beyond simply sorting dan pilih top x, bottom y", karena contohnya kalau uniform jadi ga ada rich dan poor

\begin{acknowledgments}

TBD

\end{acknowledgments}

\bibliography{mybibliography}

\end{document}