\section{Discussion}

% add to discussion 
% - Wikidata = entity-centric knowledge graph, dimana bentuknya (s,p,o) dengan s sebagai subject of interest. jadi memang expected bahwa outgoing banyak 
% sedangkan incoming mostly justru 0.
% - sedrastis ini perbedaan incoming vs outgoing. incoming link is underestimated. given o, what is the (s, p)
% - kasih contoh: 2 entitas yang secara outgoing mirip, tapi incoming-nya jomplang (1 kaya 1 miskin)
\paragraph{Generality of Framework.}
The framework that is proposed in this study is applicable to any kind of knowledge graph. However, the library that we built for experimentation is specifically for Wikidata, because the query service, structure, and response is very unique for each knowledge graph. If further experimentation is to be conducted for other knowledge graph, then the library needs to be modified.

\paragraph{Scalability.}
% In the Wikidata use case, we performed the analysis on human-related and non-human classes based on the filters that we specified. While conducting analysis on the whole human class (that is, all entites \(?s\) that satisfy the triple \((?s, instanceOf (P31), human (Q5))\) will be insightful, using smaller classes are not without any reason. The reason is the limitation of Wikidata Query Service itself, particularly on query time. The largest class that we get in a single query is the class of American politician with an entity count of 96507. Querying this class required 34 seconds. When we tried to query large classes, for example all politician without filtering their nationalities, we got errors like timeout or out of memory error.

In our Wikidata use case, we analyzed both human-related and non-human classes based on predefined filters. While analyzing the entire human class (i.e., all entities \(s\) that satisfy the triple 
\((s,instanceOf(P31),human(Q5)))\) would provide valuable insights, we opted for smaller classes due to practical constraints. The primary limitation comes from the Wikidata Query Service itself, particularly its restrictions on query execution time.

The largest class we successfully retrieved in a single query was "American politicians" which contained 96,507 entities and required 34 seconds for query execution. However, when attempting to query significantly larger classes—such as all politicians without filtering by nationality—we encountered system limitations, including timeouts and out-of-memory errors.

For Wikidata or other KGs developer, a potential solution to overcome these limitations is to host them on a more reliable infrastructure with optimized indexing and caching mechanisms. This would allow for faster query execution and improved scalability for large-scale data retrieval.

That said, it is important to note that the analysis performed in this study is non-transactional, meaning it does not require real-time execution. Instead, it is designed for periodic analyses (e.g., monthly or quarterly), where latency constraints and query limitations are less critical. Given this characteristic, our current approach remains viable despite the performance limitations of the Wikidata Query Service.

% - kenapa kita pakai class yang kecil? kenapa ga full human?
% - jumlah item terbanyak di dalam class = ...
% - experience tentang pengerjaan, gmn waktu eksekusi, kena timeout dll
% - Solution: host Wikidata / KGs itu di-host di service yang lebih reliable, memungkinkan query lebih cepat, dll
% - Solution: analisi yang kita lakukan ini non-transactional, ga mesti tiap detik. sifat analisis lebih ke regular analysis (monthly, quarterly, ...) sehingga latency dan limitation ini ga jadi masalah besar

\paragraph{Issue of Incoming Link.}
Wikidata is an entity-centric knowledge graph which means in the editathon efforts, the subject \(s\) is always be the subject of interest and the starting point to be edited by contributors. It is very unlikely that the opposite approach is done, that is, an object \((o)\) is given and a pair subject and property \((s, p)\) is to be searched. Not only from the contributors, the platform itself does not support the later view. This explains the phenomenon that we observed in \autoref{wealth & gini} regarding outgoing and incoming link. With existing subject-centric point of view, the number of new outgoing link introduced to the knowledge graph will grow in a much higher rate as opposed to incoming link. As a result, incoming link will be very rare, or even only present in certain entites.

\paragraph{Weight of Property}
%Content dump for weighted form

% It is intuitive that the measure of bag of properties will always give higher (or at least, equal) amount of wealth compared to the measure of set. Set of property will have an upper bound of number of unique property, while bag of property does not have any upper bound. Moreover, using bag of properties, a large number of triples having the same property may inflate the wealth substantially--though this is not necessarily a problem nor an advantage.

Set of properties might be more suitable if the main concern is the presence of properties, instead of the abundance of information it contains. We will take \textit{William Shakespeare} (Q692) in Wikidata as an example. It is reasonable if property like  \textit{date of birth} (P569) to be treated using set of properties, but for a well-known playwright and poet, we shall expect the property \textit{notable work} (P800) to incorporate all or most of his well-known works. If only few works registered in \(G\) despite he has dozens of works, then we might conclude the entity is poor. For this case, treating the property using the notion of set is not preferable because it will fail to capture the aforementioned poor condition, while blatantly using bag of properties might skew the wealth amount. Due to this, the notion of (non-)uniqueness can be extended to a weighted form. The weight is given independently for each property and can be defined in such a way that is most appropriate for the nature of the property.

% Due to this, the notion of (non-)uniqueness can be extended to a weighted form. The weight is given independently for each property and can be defined in such a way that is most appropriate for the nature of the property. Example of weight is inverse of median of property count of each entity.

% Let \(h_{p_i}\) be the weight of property \(p_i\) in graph \(G\). Let \(N_{bag}(s,G,p_i)\) be a set that comprises all pair  \((p_i,o)\), that is, property \(p_i\) and an object \(o\) that is connected to \(s\). Then \(W_{weighted}(s, G)\) is the sum of \(N_{bag}(s,G,p_i)\) multiplied by the associated weight \(h_{p_i}\).

% % \textit{how can we formalize wi to be the function of set(amount of pi in class C in G)??}

% % \[
% %     \forall p_i, h_i = f_i(....)
% % \]
% \[
%     N_{bag}(s,G,p_i) = \{(p_i, o) | (s, p_i, o) \in G\}
% \]
% \[
%     W_{weighted}(s, G) = \sum_i |N_{bag}(s,G,p_i)| * h_{p_i}
% \]

% Using \autoref{fig:wealth-weighted}, we can show how the notion of weighted wealth can be calculated. Let's define \(h_{p_1} = 1/median\) and \(h_{p_2} = 1\). For property \(p_1\), \(\{1, 2, 2, 4\}\) is the sorted amount of information contributed from \(p_1\) in each individual entity from \(s_1\) to \(s_4\), from which we get \(h_{p_1} = 1/median = 1/2\). With the above definition, \(W_{weighted}(s_1, G) = 2\), \(W_{weighted}(s_2, G) = 2\), \(W_{weighted}(s_3, G) = 3.5\), \(W_{weighted}(s_4, G) = 1.5\).

Another phenomena that should also be considered is that an entity might have several occurrences of the same property, but this property might just be `trivial'. This is similar to a document having a lot of `the' or `a' (stopwords). Conversely, an entity might just have a single occurrence of a property, but it is a non-trivial one. Perhaps, the property is a highly relevant one for the entity's class. For example, \textit{time period} (P2348) for \textit{William Shakespeare} (Q692) will be a highly relevant one for prominent poets. However, the property \textit{sibling} (P3373) might not be too relevant for Shakespeare's career. One idea to handle such trivial is to regard those porperties using the notion of set of properties--thus we care only on its existence rather than its actual count--or, we can use threshold function to set a tolerance for how much of such trivial we allow.

% =====> OR (alternative of weighted, or generalized form of Wealth based on the (non-)uniqueness of individual properties)

% =====> \(T_i\) is a multiset \(T_i = ... \) -> isinya list semua kontribusi wealth dari property \(p_i\) pada seluruh entity \(S\) pada class \(C\) di graph \(G\) -> keuntungannya disini kita jadi bisa punya fungsi konstan, sehingga bisa catter definisi set, bag, dan weighted secara bersamaan.

% Thus, the notion of (non-)uniqueness can be generalized to accommodate the bag, set, and weighted concept. Let \(s_1\), \(s_2\), ... \(s_m\) be \(m\) distinct entities in graph \(G\), which all of them collectively create a class \(C\). Let \(N_{bag}(s,G,p_i)\) be a set that comprises all pair of a particular property \(p_i\) and object, \((p_i,o)\), that is connected to \(s\). We define \(T_{C,p_i}\) a multiset consisted of the number of non-unique properties of each entity of \(C\) contributed from property \(p_i\), i.e., multiset of cardinality of \(N_{bag}(s,G,p_i)\) for all \(s\) in \(C\). For each \(p_i\), let \(f_{p_i}\) be a multivariate function with 2 arguments: \(T_{C,p_i}\) and \(N_{bag}(s,G,p_i)\). \(w_{p_i}\) is the amount of wealth of \(s\) attributed from property \(p_i\), which is calculated by function \(f_{p_i}\). Then \(W_{}(s, G)\) is the sum of \(w_{p_i}\).


% \[
%     N_{bag}(s,G,p_i) = \{(p_i,o) | (s_j, p_i, o) \in G\}
% \]
% \[
%     T_{C,p_i} = \{|N_{bag}(s,G,p_i)|\ | s \in C\}
% \]
% \[
%     w_{p_i} = f_{p_i}(T_{C,p_i}, |N_{bag}(s,G,p_i)|)
% \]
% \[
%     W_{}(s, G) = \sum_i w_{p_i}
% \]

% Using \autoref{fig:wealth-weighted}, we can show how the generalized form can be utilized. From the definition, we get \(T_{C,p_1} = \{2, 2, 5, 1\}\) and \(T_{C,p_2} = \{1, 1, 1, 1\}\). Property \(p_1\) illustrates the former issue of trivial. Here, we define $f_{p_1}(x,y) = \begin{cases}
%       x & \text{if }x \leq 1 \\
%       \frac{x}{2} + 1 & \text{if }1 < x \leq 4 \\
%       3 & \text{if }x > 4
%     \end{cases}\, $.
% For \(p_2\), we want to handle it using the notion of set of properties, thus we define it using a constant function \(f_{p_2}(x,y) = 1\).
% With the above definition, \(W_{}(s_1, G) = 3\), \(W_{}(s_2, G) = 3\), \(W_{}(s_3, G) = 4\), \(W_{}(s_4, G) = 2\).

\paragraph{Multiclass Entity and Misclassification.}
% - Bruno Darcet: 54 literal pure (1 date of birth + 53 elo rating)
% - Bruno Darcet (1st rank) & Daniel José Queraltó (2nd rank) ini dari profilnya sebenarnya bukan computer scientist. sepertinya ada issue missclassification pada occupation
% - Ada issue serupa dengan Computer Scientist. Yang terkaya Alexander Julien (Q61249179) bukan university researcher
% - Badminton Player, Yang richest (Dirk Stikker) bukan prominent badminton player tapi lebih ke politikus yang punya keterkaitan dengan badminton
% - suatu entitas bisa aja belong to multiple class (case Dirk Stikker). Bisa ada juga miss-classification (Bruno Darcet). Ada issue data quality, yang mengharuskan data cleaning
In data classification, entities may belong to multiple classes or be misclassified due to inconsistencies in data sources. For instance, the richest entity in the computer scientist class based on pure literal properties count, \textit{Bruno Darcet} (Q71055086), has a total wealth of 54, which accounts from 1 value in \textit{date of birth} (P569) and 53 values in \textit{Elo rating} (P1087). The same case is also observed in the second richest entity, \textit{Daniel José Queraltó} (Q23906907). From their profiles, we may say these 2 entities are not actually computer scientists, highlighting an occupational misclassification issue. A similar problem occurs with the classification of American researcher, where the wealthiest by pure literal count, \textit{Alexander Julien} (Q61249179), is not a university researcher. In the case of badminton players, \textit{Dirk Stikker} (Q194654), the richest individual in the class based on ID properties count, is more of a politician with badminton affiliations rather than a prominent player. These examples illustrate that an entity can belong to multiple classes, or be entirely misclassified. These classification inconsistencies reveal data quality issues, necessitating careful data querying, cleaning, and validation to ensure accuracy prior to analysis.