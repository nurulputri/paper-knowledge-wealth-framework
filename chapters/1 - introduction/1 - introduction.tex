\section{Introduction}

% To do:
% 1. American researchers: Yang punya property 1, 2, 3, dst itu mereka punya property apa?
% 2. [DONE] Outlier & sample size effect thd gini?
% 3. Cek perbedaan 2 entitas di Introduction (pakai query difference) --> + Introduction

% %% include motivating scenario

% (tambahkan 2 paragraf pembukaan)
% distinct - minus = apa yang ada di NYC tapi ga ada di Batam
% Kasih contoh:
% 1. Entitas dengan tipe sama
% 2. Tunjukkan adanya singevalue vs multivalue
% 3. Tunjukkan adanya presence vs absence
% 4. Tunjukkan adanya incoming vs outgoing (incoming itu ga kalah penting, bisa jadi outgoing rendah tapi incoming nya banyak di refer entitas lain)
% 5. Tunjukkan adanya prop type

% contoh lain:
% 1. yang incomingnya banyak
% 2. yang external ID banyak

% Poin:
% 1. ...
% 2. ...

Our study proposes a formal model of knowledge wealth in the RDF knowledge graph by seeing how the knowledge wealth of an entity class is spread over the knowledge graph using statistical measures and visualization. In addition, a comprehensive analytical framework is also contructed that would give insights about the wealth of a class, the inequality between classes, and imbalance measure of wealth within a class. To evaluate this framework, some use cases were conducted on several classes on Wikidata.

Our contributions are:
\begin{enumerate}
    \item We introduce the 3 notions of quantifying knowledge wealth for knowledge graphs, and show how they can be used to further characterize class-level knowledge wealth in knowledge graphs.
    \item We implemented the formal and insight model using Python programming language and made it accessible and usable through library.
    \item We perform a case study on Wikidata classes, showing how biases can be identified in Wikidata and how different definition of wealth impacts the imbalance level of a class.
\end{enumerate}