\subsection{Contribution of Property Type to Knowledge Wealth} \label{wealth & proptype contribution}

In this subchapter, analysis is done to see how much contribution of each property types to the knowledge wealth of Wikidata classes. The analysis is performed on 8 Wikidata classes, in which 4 of them are human-related class while the other 4 are not. The wealth of each classes is calculated using the concept of bag of properties and set of properties, for each property types (object, literal, ID, and their combinations).

Two methods of averaging are used in this analysis—global percentage and average contribution per individual entity. The first method is done by calculating the total sum of each property across all entities and then dividing it by the grand total of all properties combined, providing a holistic view of each property's overall contribution. The second method is done by first determining the percentage contribution of each property within each individual entity and then averaging these percentages across all entities, ensuring that every entity is equally represented regardless of its scale. While the global percentage method highlights absolute contributions, the average per-entity method captures relative contributions within each entity, making it more robust to variations in scale and less sensitive to outliers since every data point contributes equally to the final result.

From \autoref{table:prop-contribution-bag} and \autoref{table:prop-contribution-set}, generally, the usage of pure literal properties is lower than of object and ID. This aligns with the principle of knowledge graph to use URI for internal connections and ID for external resources.

We can say that object properties and literal properties combined represent wealth sourced internally from Wikidata, whereas ID properties originate from external sources. In general, more than half of the wealth is attributed to internal resources.

\begin{center}
    \scriptsize
    \makebox[\linewidth]{
    \begin{threeparttable}
    \captionsetup{font=small}
    \caption{Contribution of Property Type to Knowledge Wealth Using Bag of Properties}
    \label{table:prop-contribution-bag}
    \begin{tabular}{c | c c c | c} 
    
    \toprule
        Class Name & \CellWithForceBreak{\% Object \\ Bag} & \CellWithForceBreak{\% Literal \\ Bag} & \CellWithForceBreak{\% ID \\ Bag} & \CellWithForceBreak{\% Object + Literal \\ Bag} \\ [0.5ex] 
    \midrule
        American researcher & 59.55 / 65.65 & 3.32 / 2.8 & 37.12 / 31.55 & 62.88 / 68.45 \\
        American singer & 40.44 / 48.42 & 7.47 / 8.48 & 52.09 / 43.1 & 47.91 / 56.9 \\
        Badminton player & 79.34 / 78.96 & 10.14 / 11.68 & 10.52 / 9.36 & 89.48 / 90.64 \\
        Computer scientist & 59.36 / 65.93 & 4.19 / 3.71 & 36.44 / 30.36 & 63.56 / 69.64 \\
        Historical painting & 66.35 / 66.46 & 25.48 / 25.85 & 8.17 / 7.69 & 91.83 / 92.31 \\
        Memorial & 62.82 / 64.16 & 21.0 / 20.4 & 16.18 / 15.44 & 83.82 / 84.56 \\
        Sci-fi book & 62.96 / 62.62 & 14.44 / 15.78 & 22.59 / 21.6 & 77.41 / 78.4 \\
        University & 37.59 / 44.79 & 18.11 / 18.31 & 44.3 / 36.9 & 55.7 / 63.1 \\
        [1ex]
    \bottomrule
    \end{tabular}
    \begin{tablenotes}
        \scriptsize
        \item{This table shows the contribution of each property type of 8 Wikidata classes based on the notion of bag of properties. Each record has 2 values separated by "/". The first value is the contribution rate when calculated with global percentage method, and the second one is when using the average contribution per individual entity}
    \end{tablenotes}
    \end{threeparttable}
    }
\end{center}

\begin{center}
    \scriptsize
    \makebox[\linewidth]{
    \begin{threeparttable}
    \captionsetup{font=small}
    \caption{Contribution of Property Type to Knowledge Wealth Using Set of Properties}
    \label{table:prop-contribution-set}
    \begin{tabular}{c | c c c | c} 
    
    \toprule
        Class Name & \CellWithForceBreak{\% Object \\ Set} & \CellWithForceBreak{\% Literal \\ Set} & \CellWithForceBreak{\% ID \\ Set} & \CellWithForceBreak{\% Object + Literal \\ Set} \\ [0.5ex] 
    \midrule
        American researcher & 51.25 / 59.69 & 3.97 / 3.29 & 44.78 / 37.02 & 55.22 / 62.98 \\
        American singer & 33.91 / 43.58 & 8.1 / 9.18 & 57.99 / 47.24 & 42.01 / 52.76 \\
        Badminton player & 71.75 / 74.13 & 13.79 / 13.9 & 14.46 / 11.97 & 85.54 / 88.03 \\
        Computer scientist & 50.53 / 60.54 & 4.98 / 4.27 & 44.49 / 35.19 & 55.51 / 64.81 \\
        Historical painting & 60.57 / 61.91 & 28.88 / 28.61 & 10.55 / 9.48 & 89.45 / 90.52 \\
        Memorial & 60.06 / 62.01 & 22.41 / 21.59 & 17.54 / 16.4 & 82.46 / 83.6 \\
        Sci-fi book & 59.36 / 61.91 & 13.86 / 14.59 & 26.78 / 23.5 & 73.22 / 76.5 \\
        University & 33.93 / 42.74 & 17.67 / 18.32 & 48.4 / 38.95 & 51.6 / 61.05 \\
        [1ex]
    \bottomrule
    \end{tabular}
    \begin{tablenotes}
        \scriptsize
        \item{This table shows the contribution of each property type of 8 Wikidata classes based on the notion of set of properties. Each record has 2 values separated by "/". The first value is the contribution rate when calculated with global percentage method, and the second one is when using the average contribution per individual entity}
    \end{tablenotes}
    \end{threeparttable}
    }
\end{center}

Typically, an ID property has exactly one associated value because it serves as a unique identifier in external resources, preventing ambiguity. In contrast, object properties can have multiple values. For example, the computer scientist \textit{Noam Chomsky} (Q9049) has an ID property, \textit{VIAF cluster ID} (P214), with exactly one value: \textit{89803084}. At the same time, he has an object property, \textit{work location} (P937), with four values: \textit{Pennsylvania} (Q1400), \textit{Cambridge} (Q350), \textit{Tucson} (Q18575), and \textit{Massachusetts} (Q771). When using the bag of properties, \textit{VIAF cluster ID} (P214) contributes 1 to the total wealth, while \textit{work location} (P937) contributes 4. Under the set of properties, both properties contribute equally, each accounting for 1. This explains why the percentage contribution of object properties to wealth is smaller when using the set of properties compared to the bag of properties. Since the overall property count is lower in the set of properties, the percentage contribution of ID properties to wealth is correspondingly higher than in the bag of properties.