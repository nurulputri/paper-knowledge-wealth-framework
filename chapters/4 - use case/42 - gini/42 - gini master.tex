\subsection{Effect of Type of Wealth on Inequality Measure}

% \begin{center}
    \small
    \begin{threeparttable}
    \caption{Gini Bag-Set}
    \label{tab:gini bag-set}
    \begin{tabular}{c c c} 
    
    \toprule
        Class Name & Gini Bag & Gini Set \\ [0.5ex] 
    \midrule
        Historical painting & 0.22 & 0.14 \\
        University & 0.44 & 0.41 \\
        Sci-fi book & 0.32 & 0.22 \\
        Memorial & 0.22 & 0.18 \\
        American researcher & 0.33 & 0.31 \\
        American singer & 0.40 & 0.39 \\
        Badminton player & 0.29 & 0.14 \\
        Computer scientist & 0.41 & 0.37 \\
        [1ex]
    \bottomrule
    \end{tabular}
    \begin{tablenotes}
        \footnotesize
        This table shows the measures of gini coefficient of knowledge wealth based on the (non-)uniqueness of individual properties of 8 Wikidata classes
    \end{tablenotes}
    \end{threeparttable}
\end{center}
% \begin{center}
    \small
    \begin{threeparttable}
    \caption{Gini Object-Literal-ID}
    \label{tab:gini proptype}
    \begin{tabular}{c c c c} 
    
    \toprule
        Class Name & Gini Object & Gini Literal & Gini ID \\ [0.5ex] 
    \midrule
        American researcher & 0.26 & 0.50 & 0.50 \\
        American singer & 0.29 & 0.50 & 0.53 \\
        Badminton player & 0.30 & 0.36 & 0.60 \\
        Computer scientist & 0.36 & 0.54 & 0.56 \\
        Historical painting & 0.25 & 0.27 & 0.44 \\
        Memorial & 0.20 & 0.30 & 0.40 \\
        Sci-fi book & 0.35 & 0.34 & 0.42 \\
        University & 0.40 & 0.49 & 0.53 \\
        [1ex]
    \bottomrule
    \end{tabular}
    \begin{tablenotes}
        \footnotesize
        This table shows the measures of gini coefficient of knowledge wealth based on types of property of 8 Wikidata classes
    \end{tablenotes}
    \end{threeparttable}
\end{center}
% \begin{center}
    \small
    \begin{threeparttable}
    \caption{Gini Outgoing-Incoming}
    \label{tab:gini outgoing-incoming}
    \begin{tabular}{c c c} 
    
    \toprule
        Class Name & Gini Outgoing & Gini Incoming \\ [0.5ex] 
    \midrule
        Historical painting & 0.22 & 0.86 \\
        University & 0.44 & 0.91 \\
        Sci-fi book & 0.32 & 0.82 \\
        Memorial & 0.22 & 0.99 \\
        American researcher & 0.33 & 0.77 \\
        American singer & 0.40 & 0.82 \\
        Badminton player & 0.29 & 0.68 \\
        Computer scientist & 0.41 & 0.81 \\
        [1ex]
    \bottomrule
    \end{tabular}
    \begin{tablenotes}
        \footnotesize
        This table shows the measures of gini coefficient of knowledge wealth based on the direction of link of 8 Wikidata classes
    \end{tablenotes}
    \end{threeparttable}
\end{center}

In this subchapter, analysis is done to see how each wealth type affects the level of inequality of Wikidata classes. There are 2 ways this is done--quantitatively using Gini coefficient and qualitatively using Lorenz curve. The analysis is performed on 8 Wikidata classes, in which 4 of them are human-related class while the other 4 are not.

When looking at the notion of wealth using the characteristics of (non-)uniqueness of individual properties, it is intuitive that the measure of bag of properties will always give higher (or at least, equal) amount of wealth compared to the measure of set. Set of property will have an upper bound of number of unique property, while bag of properties does not have any upper bound. Moreover, using bag of properties, a large number of triples having the same property may inflate the wealth substantially--though this is not necessarily a problem nor an advantage. This characteristics has a direct impact on inequality measure and it well depicted on the value of Gini coefficient. From \autoref{table:gini-coef}, in all clasess, the Gini coefficient using bag of properties is always higher than of set of properties.

- object, literal, ID: object tends to have lower inequality

- outgoing, incoming: incoming shows very high Gini, this is because it is harder --> show the lorenz curve

add to discussion 
- Wikidata = entity-centric knowledge graph, dimana bentuknya (s,p,o) dengan s sebagai subject of interest. jadi memang expected bahwa outgoing banyak 
sedangkan incoming mostly justru 0.
- sedrastis ini perbedaan incoming vs outgoing. incoming link is underestimated. given o, what is the (s, p)
- kasih contoh: 2 entitas yang secara outgoing mirip, tapi incoming-nya jomplang (1 kaya 1 miskin)

\begin{center}
    \small
    \begin{threeparttable}
    \caption{Knowledge Wealth Type on Gini Coefficient}
    \label{table:gini-coef}
    \begin{tabular}{c | c c | c c c | c c} 
    
    \toprule
        Class Name & \CellWithForceBreak{Gini \\ Bag} & \CellWithForceBreak{Gini \\ Set} & \CellWithForceBreak{Gini \\ Object} & \CellWithForceBreak{Gini \\ Literal} & \CellWithForceBreak{Gini \\ ID} & \CellWithForceBreak{Gini \\ Outgoing} & \CellWithForceBreak{Gini \\ Incoming} \\ [0.5ex] 
    \midrule
        American researcher & 0.33 & 0.31 & 0.25 & 0.27 & 0.44 & 0.33 & 0.77 \\
        American singer & 0.40 & 0.39 & 0.20 & 0.30 & 0.40 & 0.40 & 0.82 \\
        Badminton player & 0.29 & 0.14 & 0.35 & 0.34 & 0.42 & 0.29 & 0.68 \\
        Computer scientist & 0.41 & 0.37 & 0.40 & 0.49 & 0.53 & 0.41 & 0.81 \\
        Memorial & 0.22 & 0.18 & 0.36 & 0.54 & 0.56 & 0.22 & 0.99 \\
        Historical painting & 0.22 & 0.14 & 0.26 & 0.50 & 0.50 & 0.22 & 0.86 \\
        Sci-fi book & 0.32 & 0.22 & 0.30 & 0.36 & 0.60 & 0.32 & 0.82 \\
        University & 0.44 & 0.41 & 0.29 & 0.50 & 0.53 & 0.44 & 0.91 \\
        [1ex]
    \bottomrule
    \end{tabular}
    \begin{tablenotes}
        \footnotesize
        This table shows the comparison of Gini coefficient of 8 Wikidata classes
    \end{tablenotes}
    \end{threeparttable}
\end{center}